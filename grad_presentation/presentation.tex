\documentclass[hyperref=unicode, aspectratio=169]{beamer}
\usepackage[utf8]{inputenc}
\usepackage[T2A]{fontenc}
\usepackage[russian]{babel}
\usepackage{csquotes}
\usepackage{tikzit}
\usepackage[justification=centering]{caption}
\usepackage{mathabx}
\usepackage{bookmark}
\usepackage{makecell}
\usepackage{listings}
\usepackage{flowchart}
\usepackage{subcaption}
\usepackage{caption}

\captionsetup[figure]{labelformat=empty}


\title[]{Жадные алгоритмы для построения многопроцессорного списочного расписания}
\usetheme{Madrid}
\author[]{Савицкий Илья\\Научный руководитель: к.т.н. доцент Костенко Валерий Алексеевич}
% \medskip
% \texttt{Научный руководитель: к.т.н. доцент Костенко Валерий Алексеевич}
\date{27 апреля 2023 г.}
\logo{\includegraphics[width=1.5cm]{imgs/asvk-logo.png}}


\definecolor{ASVKaccent}{rgb}{0.20392,0.02745,0.34509}
\usecolortheme[named=ASVKaccent]{structure}

\input{block-schemas.tikzstyles}

\begin{document}

\begin{frame}
    \titlepage
\end{frame}

\section{Цель и задачи дипломной работы}
\begin{frame}
    \frametitle{Цели и задачи курсовой}
    Для достижения указанной цели требуется:
    \begin{enumerate}
        \item Провести обзор алгоритмов построения списочных расписаний с целью выявления жадных критериев и схем ограниченного перебора которые могут быть модифицированы для решения данной задачи.
        \item Разработать алгоритм.
        \item Реализовать алгоритм.
        \item Провести исследование свойств алгоритма.
    \end{enumerate}
\end{frame}

\section{Описание прикладной задачи}
\begin{frame}
    \frametitle{Постановка задачи}
    \begin{columns}
        \begin{column}{0.60\textwidth}
            \begin{enumerate}
                \item Ориентированный граф работ $G$ без циклов, в котором дуги - зависимости по данным, а вершины - задания. Вершин $n$, дуг $m$
                \item Вычислительная система, состоящая из $p$ различных процессоров.
                \item Матрица $C_{ij}$ длительности выполнения работ на процессорах, $i=1 \dots n, j=1 \dots p$. Каждая строка этой матрицы - длины выполнения $n$-й задачи на $p$ процессорах. 
                \item Матрица $D_{kl}$ передач данных между процессорами, $k=1 \dots p, l = 1 \dots p, D_{kk} = 0$. $D_{ij}$-й элемент этой матрицы - время передачи данных между процессорами $i$ и $j$.
            \end{enumerate}
        \end{column}
        \begin{column}{0.40\textwidth}
            \begin{figure}
                \ctikzfig{graph_schema}
                \captionsetup{labelformat=empty}
                \caption{Граф потока данных}
            \end{figure}
        \end{column}
    \end{columns}
\end{frame}

\begin{frame}
    \frametitle{Расписание}
    Расписание программы определено, если определены
    \begin{enumerate}
        \item Множества процессоров и работ
        \item Привязка
        \item Порядок
    \end{enumerate}
    \par
    \textbf{Привязка} - всюду определенная на множестве работ функция, которая задает распределение работ по процессорам.
    \par
    \textbf{Порядок} задает ограничения на последовательность выполнения работ и является отношением частичного порядка, удовлетворяющим условиям ацикличности и транзитивности. Отношение порядка на множестве работ, распределенных на один процессор, является отношением полного порядка.
\end{frame}

\begin{frame}
    \frametitle{Постановка задачи}
    \begin{columns}
        \begin{column}{0.7\textwidth}
            Требуется:
            \begin{enumerate}
                \item Построить расписание $HP$, то есть для $i$-й работы определить время начала ее выполнения $s_i$ и процессор $p_i$ на котором она будет выполняться;
                \item Минимизируемый критерий: время завершения выполнения расписания.
            \end{enumerate}
        \end{column}
        \begin{column}{0.3\textwidth}
            \begin{figure}
                \tiny
                \ctikzfig{figures/schedule-time-diagram}
                \captionsetup{labelformat=empty}
                \caption{\small Представление расписания в виде временной диаграммы}
            \end{figure}
        \end{column}
    \end{columns}
\end{frame}

\begin{frame}
    \frametitle{Модель расписания}
    Множество корректных расписаний $HP$ задается набором ограничений:
    \begin{itemize}
        \item В расписании не допустимы прерывания;
        \item Интервалы выполнения работ не пересекаются;
        \item Каждая работа назначена на процессор;
        \item Любую работу обслуживает один процессор;
        \item Частичный порядок, заданный графом зависимостей $G$, сохранен в $HP: G \subset G_{HP}^T$, где $G_{HP}^T$ - транзитивное замыкание отношения $G_{HP}$.
    \end{itemize}
\end{frame}

\begin{frame}
    \frametitle{Дополнительные ограничения}
    \begin{enumerate}
        \item Задача с однородными процессорами (длительность выполнения работы не зависит от того, на каком процессоре она выполняется) и дополнительными ограничениями на количество передач:
              \begin{itemize}
                  \item $CR = \frac{m_{ip}}{m}$, где $m_{ip}$ - количество передач данных между работами на каждый процессор
              \end{itemize}
        \item Задача без дополнительных ограничений.
    \end{enumerate}
\end{frame}


\section{Выводы по обзору}
\newcolumntype{Y}{>{\centering\arraybackslash}X}

\begin{frame}
    \frametitle{Обзор предметной области}
    Проведен обзор детерминированных алгоритмов, которые возможно модифицировать под поставленную задачу и имеют хорошую возможность масштабирования.
    \begin{table}
        \begin{tabularx}{\textwidth}{  c | Y | Y }
            Название алгоритма                   & Возможность модификации алгоритма & Возможность масштабирования алгоритма \\
            \hline
            Метод ветвей и границ                & \ding{51}                         & \ding{55}                             \\
            Метод динамического программирования & \ding{51}                         & \ding{55}                             \\
            Алгоритм поиска максимального потока & \ding{55}                         & \ding{51}                             \\
            Жадные алгоритмы                     & \ding{51}                         & \ding{51}                             \\
        \end{tabularx}
    \end{table}

    % {
    %     \small
    %     \begin{tabular}{ c | c | c | c  }
    %         \makecell{Название          \\алгоритма} & Рандомизированность & Итерационный & \makecell{Возможность \\ масштабирования} \\
    %         \hline
    %         \makecell{Генетические      \\алгоритмы} & Рандомный & Итерационный & +/- \\
    %         \makecell{Алгоритм имитации \\отжига} & Рандомный & Итерационный & + \\
    %         \makecell{Муравьиные        \\алгоритмы} & Рандомный & Итерационный & - \\
    %         \makecell{Жадные стратегии  \\и ограниченный перебор} & Детерминированный & Конструктивный & + \\
    %     \end{tabular}
    % }
\end{frame}

\section{Описание предложенного алгоритма}
% \begin{frame}
%     \frametitle{Дополнительные обозначения}
%     \begin{enumerate}
%         \item $D= \left( d_1, d_2, \dots, d_l \right)$, где $l$ - количество вершин, доступных для добавления.
%         \item $\left( s_i, p_i \right)$ - достаточное количество информации для размещения работы в расписании. Установка соотношения между работой $t$ и парой $\left( s_i, p_i \right)$ и есть построение расписания
%     \end{enumerate}
%     \hrule
%     \vspace{2pt}
%     Жадные критерии
%     \begin{enumerate}
%         \item $GC1$ - критерий, используемый в выборе работы на постановку
%         \item $GC2$ - критерий, используемый в выборе места постановки работы
%     \end{enumerate}
%     \hrule
%     \vspace{2pt}
%     EDF-эвристика.
% \end{frame}


\begin{frame}
    \frametitle{Общая схема жадных алгоритмов построения расписания}
    {
        \small
        \begin{tikzpicture}
            \node (start) at (0, 0) [draw, terminal] {Начало};
            \node (decision) at (0, -2) [draw, decision, align=center] {Все работы добавлены \\ в расписание?};
            \node (finish) at (-4, -3) [draw, terminal] {Конец};
            \node (choose_task) at (0, -4.1) [draw, process] {Выбрать следующую работу для постановки};
            \node (choose_proc) at (0, -5.5) [draw, process] {Выбрать процессор для работы};
            \node (add_task) at (6, -5.5) [draw, process] {Поставить работу на процессор};

            \draw[thick, ->] (start) -- (decision);
            \draw[thick, ->] (decision) -- node[right]{Нет} (choose_task);
            \draw[thick, ->] (choose_task) -- (choose_proc);
            \draw[thick, ->] (choose_proc) -- (add_task);
            \draw[thick, ->] (add_task) |- (decision);
            \draw[thick, ->] (decision) -| node[above]{Да} (finish);
        \end{tikzpicture}
    }

\end{frame}

\begin{frame}
    \frametitle{Жадный алгоритм с выбором по числу потомков}
    \begin{enumerate}
        \item Выбор следующей работы на постановку - критерий $GC1$
        \item Выбор процессора для работы
              \begin{itemize}
                  \item Для $CR$ - из изначально заданного распределения
                  \item Для $NO$ - по критерию $GC2$
              \end{itemize}
    \end{enumerate}
    Зададим множество доступных для добавления вершин $D= \left( d_1, d_2, \dots, d_l \right)$, где $l$ - количество вершин, доступных для добавления.
    \begin{columns}
        \begin{column}{0.6\textwidth}
            \textbf{Критерий $GC1$:} \\
            Из множества $\color{red}D$ выбирается работa по критерию $GC1$ максимальности количества потомков у вершины. \\
            \textbf{Критерий $GC2$:} \\
            Работа ставится на процессор, на котором время завершения работы будет минимальным.

        \end{column}
        \begin{column}{0.4\textwidth}
            \ctikzfig{max_children}
        \end{column}
    \end{columns}
\end{frame}

\begin{frame}
    \frametitle{Алгоритм постановки работы на процессор}
    \begin{columns}
        \begin{column}{0.55\textwidth}
            При постановке требуется найти такое минимальное время $t$, чтобы
            \begin{enumerate}
                \item Все передачи данных завершились до $t$;
                \item Существует интервал простоя длительности не меньший времени выполнения работы, начинающийся в $t$.
            \end{enumerate}
        \end{column}
        \begin{column}{0.45\textwidth}
            {
                \tiny
                \ctikzfig{schedule-time-diagram-new-3}
            }
            {
                \tiny
                \ctikzfig{schedule-time-diagram-new-2}
            }
        \end{column}
    \end{columns}
\end{frame}

\begin{frame}
    \frametitle{Жадный алгоритм с фиктивными директивными сроками}
    \begin{enumerate}
        \item Выбор следующей работы на постановку - в порядке возрастания фиктивных директивных сроков;
        \item Выбор процессора для работы:
              \begin{itemize}
                  \item Для $CR$ - из изначально заданного распределения
                  \item Для $NO$ - по критерию $GC2$
              \end{itemize}
    \end{enumerate}
    % Пусть \textbf{длина пути} - сумма всех задержек передач данных и времен выполнения работ на процессорах.
    % Пусть директивный срок всего расписания $d$, а $p_A$ - длина длиннейшего пути от работы $A$ до работы $S$ такой, что у $S$ нет потомков. Тогда директивный срок $d_A$ вершины $A$ равен $d_A - p$.
    % \begin{figure*}
    \begin{columns}
        \begin{column}{0.5\textwidth}
            \ctikzfig{edf}
        \end{column}
        \begin{column}{0.5\textwidth}
            Распространение директивных сроков по графу потока управления.\\Все межпроцессорные передачи равны 2. Только $x_2$ и $x_5$ находятся на разных процессорах.
        \end{column}
    \end{columns}
    % \captionof{figure}{Распространение директивных сроков по графу потока управления.\\Все межпроцессорные передачи равны 2. Только $x_2$ и $x_5$ находятся на разных\\процессорах.}
    % \end{figure*}
\end{frame}



\section{Результаты исследования}
\begin{frame}
    \frametitle{Точность полученного расписания}
    \begin{columns}
        \begin{column}{0.5\textwidth}
            \begin{figure}
                \includegraphics[width=\textwidth]{imgs/relative_score.png}
            \end{figure}
        \end{column}
        \begin{column}{0.5\textwidth}
            \begin{table}
                \caption*{Наборы исходных данных, используемых в тестировании}
                \begin{tabular}{c | c | c | c}
                    Граф & Вершин & Процессоров & Передач \\
                    \hline
                    1    & 126    & 4           & 716     \\
                    2    & 417    & 8           & 2367    \\
                    3    & 408    & 8           & 8763    \\
                    4    & 296    & 8           & 395     \\
                    5    & 93     & 4           & 92      \\
                \end{tabular}
            \end{table}
            % \begin{enumerate}
            %     \item 126 вершин, 4 процессора, 716 передач данных
            %     \item 417 вершин, 8 процессоров, 2367 передач данных
            %     \item 408 вершин, 8 процессоров, 8763 передач данных
            %     \item 396 вершин, 8 процессоров, 395 передачи данных
            %     \item 93 вершины, 4 процессора, 92 передачи данных
            % \end{enumerate}
        \end{column}
    \end{columns}

    \begin{equation*}
        relative\_score = \frac{\text{время полученного расписания}}{\text{время оптимального расписания}} - 1
    \end{equation*}
\end{frame}

\begin{frame}
    \frametitle{Время выполнения программы}
    \begin{figure}
        \includegraphics[width=0.7\textwidth]{imgs/times.png}
    \end{figure}
\end{frame}

\section{Полученные рещультаты}
\begin{frame}
    \frametitle{Текущие результаты}
    Реализовано:
    \begin{enumerate}
        \item Проведен обзор алгоритмов построения списочных расписаний. Цель обзора: выявление жадных критериев и схем ограниченного перебора, которые могут быть модифицированы для решения данной задачи.
        \item Разработан и ревлизован алгоритм, основанный на сочетании жадных стратегий и ограниченного перебора.
        \item Подобраны оптимальные параметры алгоритма.
        \item Проведено детальное исследование свойств алгоритма. 
    \end{enumerate}
\end{frame}

\begin{frame}
    \frametitle{Проблема проверки алгоритма на данных с известным оптимумом}
    \begin{figure}
        \includegraphics[width=0.6\textwidth]{imgs/balanced-schedule.jpg}
    \end{figure}
\end{frame}

\begin{frame}
    \frametitle{Точность полученного расписания. CR и NO}
    \begin{figure}
        \begin{subfigure}{0.49\textwidth}
            \includegraphics[width=\textwidth]{imgs/layered_class_1/CR_EDF/gr_amalgamated.png}
            \caption{При постановке CR}
        \end{subfigure}
        \begin{subfigure}{0.49\textwidth}
            \includegraphics[width=\textwidth]{imgs/layered_class_1/NO_EDF/gr_amalgamated.png}
            \caption{При постановке NO}
        \end{subfigure}
        \caption{Отношение длительности работы алгоритма с фиктивными директивными сроками к длительности\\работы жадного алгоритма на данных, основанных на слоистых графах}
    \end{figure}
\end{frame}

\begin{frame}
    \frametitle{Точность полученного расписания. CR и NO}
    \begin{figure}
        \begin{subfigure}{0.49\textwidth}
            \includegraphics[width=\textwidth]{imgs/unbalanced/CR_EDF/gr_amalgamated.png}
            \caption{При постановке CR}
        \end{subfigure}
        \begin{subfigure}{0.49\textwidth}
            \includegraphics[width=\textwidth]{imgs/unbalanced/NO_EDF/gr_amalgamated.png}
            \caption{При постановке NO}
        \end{subfigure}
        \caption{Отношение длительности работы алгоритма с фиктивными директивными сроками к длительности\\работы жадного алгоритма на данных, основанных на неоднородных процессорах}
    \end{figure}
\end{frame}

\begin{frame}
    \frametitle{Время выполнения программы. CR и NO.}
    \begin{figure}
        \begin{subfigure}{0.49\textwidth}
            \includegraphics[width=\textwidth]{imgs/ideal_1/CR/tr_graph.png}
            \caption{Жадный алгоритм с выбором по числу потомков}
        \end{subfigure}
        \begin{subfigure}{0.49\textwidth}
            \includegraphics[width=\textwidth]{imgs/ideal_1/CR_EDF/tr_graph.png}
            \caption{Жадный алгоритм с фиктивными директивными сроками}
        \end{subfigure}
        \caption{Время выполнения алгоритма на данных с известным оптимумом, в секундах}
    \end{figure}
\end{frame}


\end{document}