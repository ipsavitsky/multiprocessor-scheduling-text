\begin{frame}
    \frametitle{Постановка задачи}
    \begin{columns}
        \begin{column}{0.60\textwidth}
            \begin{enumerate}
                \item Граф потока управления $G$ без циклов, в котором дуги - зависимости по данным, а вершины - задания. Вершин $n$, дуг $m$
                \item Вычислительная система, состоящая из $p$ процессоров.
                \item Матрица $C_{n \times p}$ времени выполнения работ на процессорах. Каждая строка этой матрицы - длины выполнения $n$-й задачи на $p$ процессорах. 
                % \item Матрица $D_{kl}$ передач данных между процессорами, $k=1 \dots p, l = 1 \dots p, D_{kk} = 0$. $D_{ij}$-й элемент этой матрицы - время передачи данных между процессорами $i$ и $j$.
                \item Время $d$, затрачиваемое на межпроцессорную передачу.
            \end{enumerate}
        \end{column}
        \begin{column}{0.40\textwidth}
            \begin{figure}
                \ctikzfig{graph_schema}
                \captionsetup{labelformat=empty}
                \caption{Граф потока данных}
            \end{figure}
        \end{column}
    \end{columns}
\end{frame}

% \begin{frame}
%     \frametitle{Расписание}
%     Расписание программы определено, если определены
%     \begin{enumerate}
%         \item Множества процессоров и работ
%         \item Привязка
%         \item Порядок
%     \end{enumerate}
%     \par
%     \textbf{Привязка} - всюду определенная на множестве работ функция, которая задает распределение работ по процессорам.
%     \par
%     \textbf{Порядок} задает ограничения на последовательность выполнения работ и является отношением частичного порядка, удовлетворяющим условиям ацикличности и транзитивности. Отношение порядка на множестве работ, распределенных на один процессор, является отношением полного порядка.
% \end{frame}

\begin{frame}
    \frametitle{Постановка задачи}
    \begin{columns}
        \begin{column}{0.7\textwidth}
            Требуется:
            \begin{enumerate}
                \item Построить расписание $HP$, то есть для $i$-й работы определить время начала ее выполнения $s_i$ и процессор $p_i$ на котором она будет выполняться;
                \item Минимизируемый критерий: время завершения выполнения расписания.
            \end{enumerate}
        \end{column}
        \begin{column}{0.3\textwidth}
            \begin{figure}
                \tiny
                \ctikzfig{figures/schedule-time-diagram}
                \captionsetup{labelformat=empty}
                % \caption{\small Представление расписания в виде временной диаграммы}
            \end{figure}
        \end{column}
    \end{columns}
\end{frame}

\begin{frame}
    \frametitle{Модель расписания}
    Множество корректных расписаний $HP$ задается набором ограничений:
    \begin{itemize}
        \item В расписании не допустимы прерывания;
        \item Интервалы выполнения работ не пересекаются;
        \item Каждая работа назначена на процессор;
        \item Любую работу обслуживает один процессор;
        \item Частичный порядок, заданный графом потока управления $G$, сохранен в $HP$.
    \end{itemize}
\end{frame}

\begin{frame}
    \frametitle{Дополнительные ограничения}
    \begin{enumerate}
        \item Задача без дополнительных ограничений.
        \item Задача с дополнительным ограничением на количество передач:
              \begin{itemize}
                  \item $CR = \frac{m_{ip}}{m}$, где $m_{ip}$ - количество межпроцессорных передач в расписании.
              \end{itemize}
    \end{enumerate}
\end{frame}
