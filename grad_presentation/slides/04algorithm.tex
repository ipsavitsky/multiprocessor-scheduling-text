% \begin{frame}
%     \frametitle{Дополнительные обозначения}
%     \begin{enumerate}
%         \item $D= \left( d_1, d_2, \dots, d_l \right)$, где $l$ - количество вершин, доступных для добавления.
%         \item $\left( s_i, p_i \right)$ - достаточное количество информации для размещения работы в расписании. Установка соотношения между работой $t$ и парой $\left( s_i, p_i \right)$ и есть построение расписания
%     \end{enumerate}
%     \hrule
%     \vspace{2pt}
%     Жадные критерии
%     \begin{enumerate}
%         \item $GC1$ - критерий, используемый в выборе работы на постановку
%         \item $GC2$ - критерий, используемый в выборе места постановки работы
%     \end{enumerate}
%     \hrule
%     \vspace{2pt}
%     EDF-эвристика.
% \end{frame}


\begin{frame}
    \frametitle{Общая схема жадных алгоритмов построения расписания}
    {
        \small
        \begin{tikzpicture}
            \node (start) at (0, 0) [draw, terminal] {Начало};
            \node (decision) at (0, -2) [draw, decision, align=center] {Все работы добавлены \\ в расписание?};
            \node (finish) at (-4, -3) [draw, terminal] {Конец};
            \node (choose_task) at (0, -4.1) [draw, process] {Выбрать следующую работу для постановки};
            \node (choose_proc) at (0, -5.5) [draw, process] {Выбрать процессор для работы};
            \node (add_task) at (6, -5.5) [draw, process] {Поставить работу на процессор};

            \draw[thick, ->] (start) -- (decision);
            \draw[thick, ->] (decision) -- node[right]{Нет} (choose_task);
            \draw[thick, ->] (choose_task) -- (choose_proc);
            \draw[thick, ->] (choose_proc) -- (add_task);
            \draw[thick, ->] (add_task) |- (decision);
            \draw[thick, ->] (decision) -| node[above]{Да} (finish);
        \end{tikzpicture}
    }

\end{frame}

\begin{frame}
    \frametitle{Жадный алгоритм с выбором по числу потомков}
    \begin{enumerate}
        \item Выбор следующей работы на постановку - критерий $GC1$
        \item Выбор процессора для работы
              \begin{itemize}
                  \item Для $CR$ - из изначально заданного распределения
                  \item Для $NO$ - по критерию $GC2$
              \end{itemize}
    \end{enumerate}
    Зададим множество доступных для добавления вершин $D= \left( d_1, d_2, \dots, d_l \right)$, где $l$ - количество вершин, доступных для добавления.
    \begin{columns}
        \begin{column}{0.6\textwidth}
            \textbf{Критерий $GC1$:} \\
            Из множества $\color{red}D$ выбирается работa по критерию $GC1$ максимальности количества потомков у вершины. \\
            \textbf{Критерий $GC2$:} \\
            Работа ставится на процессор, на котором время завершения работы будет минимальным.

        \end{column}
        \begin{column}{0.4\textwidth}
            \ctikzfig{max_children}
        \end{column}
    \end{columns}
\end{frame}

\begin{frame}
    \frametitle{Алгоритм постановки работы на процессор}
    \begin{columns}
        \begin{column}{0.55\textwidth}
            При постановке требуется найти такое минимальное время $t$, чтобы
            \begin{enumerate}
                \item Все передачи данных завершились до $t$;
                \item Существует интервал простоя длительности не меньший времени выполнения работы, начинающийся в $t$.
            \end{enumerate}
        \end{column}
        \begin{column}{0.45\textwidth}
            {
                \tiny
                \ctikzfig{schedule-time-diagram-new-3}
            }
            {
                \tiny
                \ctikzfig{schedule-time-diagram-new-2}
            }
        \end{column}
    \end{columns}
\end{frame}

\begin{frame}
    \frametitle{Жадный алгоритм с фиктивными директивными сроками}
    \begin{enumerate}
        \item Выбор следующей работы на постановку - в порядке возрастания фиктивных директивных сроков;
        \item Выбор процессора для работы:
              \begin{itemize}
                  \item Для $CR$ - из изначально заданного распределения
                  \item Для $NO$ - по критерию $GC2$
              \end{itemize}
    \end{enumerate}
    % Пусть \textbf{длина пути} - сумма всех задержек передач данных и времен выполнения работ на процессорах.
    % Пусть директивный срок всего расписания $d$, а $p_A$ - длина длиннейшего пути от работы $A$ до работы $S$ такой, что у $S$ нет потомков. Тогда директивный срок $d_A$ вершины $A$ равен $d_A - p$.
    % \begin{figure*}
    \begin{columns}
        \begin{column}{0.5\textwidth}
            \ctikzfig{edf}
        \end{column}
        \begin{column}{0.5\textwidth}
            Распространение директивных сроков по графу потока управления.\\Все межпроцессорные передачи равны 2. Только $x_2$ и $x_5$ находятся на разных процессорах.
        \end{column}
    \end{columns}
    % \captionof{figure}{Распространение директивных сроков по графу потока управления.\\Все межпроцессорные передачи равны 2. Только $x_2$ и $x_5$ находятся на разных\\процессорах.}
    % \end{figure*}
\end{frame}

