\subsection{Цели и методика экспериментального исследования}
Целями экспериментального исследования было поставлено исследование:
\begin{itemize}
    \item Качество решений, получаемых алгоритмами.
    \item Время работы алгоритма.
\end{itemize}

Для проведения экспериментов было сгенерированно 3 набора данных:
\begin{enumerate}
    \item \label{item:known_opt_data} Набор данных с известных оптимумом, для постановки задачи с дополнительным ограничением $CR$
    \item Набор данных, основанных на слоистых графах, без известного оптимума, но с однородными процессорами. Данные из пункта \ref{item:known_opt_data} имеют свойство идеально сбалансированного распределения, то есть от METIS всегда построит распределение, близкое к распределению идеального расписания. Чтобы проверить, как ведут себя алгоритмы на данных без такого свойства, были добавлено исследование на данных слоистых графов. Данный набор так же используется для исследований алгоритма для постановки задачи с дополнительным ограничением $CR$.
    \item Набор данных, основанный на слоистых графах, без известного оптимума, но с неоднородными процессорами. Используется для постановки задачи без дополнительных ограничений
\end{enumerate}

Схема генерации слоистых графов описана в \cite{Canon_2019}.

\subsection{Экспериментальный стенд}

Эксперименты были проведены на машине, обладающей следующими характеристиками:
\begin{itemize}
    \item CPU Intel Xeon E5-2605 v4, 2.2Ггц
    \item 62Гб оперативной памяти
\end{itemize}

\subsection{Исследование качества решений}

\subsubsection{Жадный алгоритм с выбором по числу потомков}

\paragraph{Постановка задачи с ограничением на количество передач}

\begin{figure}[!htbp]
    \centering
    \begin{subfigure}{0.49\textwidth}
        \includegraphics[width=\textwidth]{imgs/ideal_1/CR/th.png}
        \caption{Тепловая карта}
        \label{fig:CR-GC1-times-heatmap}
    \end{subfigure}
    \hfill
    \begin{subfigure}{0.49\textwidth}
        \includegraphics[width=\textwidth]{imgs/ideal_1/CR/gr_amalgamated.png}
        \caption{Сводный график}   
        \label{fig:CR-GC1-times-compiled} 
    \end{subfigure}
    \caption{Отношение времени выполнения расписания к оптимальному времени выполнения на данных с известным оптимумом на постановке $CR$}
\end{figure}

На рисунках \ref{fig:CR-GC1-times-heatmap} и \ref{fig:CR-GC1-times-compiled} показано качество решений, генерируемых жадных алгоритмом с выбором по числу потомков на данных с известным оптимумом. Цветом на рисунке \ref{fig:CR-GC1-times-heatmap} и значением на оси $Oy$ на рисунке \ref{fig:CR-GC1-times-compiled} показано отношение длительности расписания, построенного алгоритмом к длительности оптимального расписания. Значения всегда больше 1, чем меньше, тем лучше.

Точность алгоритма повышается с увеличением количества работ и уменьшается с повышением количества процессоров в системе. 

% \begin{figure}[!htbp]
%     \centering
%     \includegraphics[width=\textwidth]{imgs/ideal_1/CR/cr_3d.png}
%     \caption{12345}
% \end{figure}

% \begin{figure}[!htbp]
%     \centering
%     \includegraphics[width=\textwidth]{imgs/layered_class_1/CR/cr_3d.png}
%     \caption{12345}    
% \end{figure}

% \begin{figure}[!htbp]
%     \centering
%     \includegraphics[width=\textwidth]{imgs/unbalanced/CR/cr_3d.png}
%     \caption{12345}
% \end{figure}

\paragraph{Постановка задачи без ограничений}

\begin{figure}[!htbp]
    \centering
    \begin{subfigure}{0.49\textwidth}
        \includegraphics[width=\textwidth]{imgs/ideal_1/NO/th.png}
        \caption{Тепловая карта}   
        \label{fig:NO-GC1-times-heatmap}
    \end{subfigure}
    \hfill
    \begin{subfigure}{0.49\textwidth}
        \includegraphics[width=\textwidth]{imgs/ideal_1/NO/gr_amalgamated.png}
        \caption{Сводный график}   
        \label{fig:NO-GC1-times-compiled} 
    \end{subfigure}
    \caption{Отношение времени выполнения расписания к оптимальному времени выполнения на данных с известным оптимумом на постановке $NO$}
\end{figure}

На рисунках \ref{fig:NO-GC1-times-heatmap} и \ref{fig:NO-GC1-times-compiled} показано качество решений, генерируемых жадных алгоритмом с выбором по числу потомков на данных с известным оптимумом. Цветом на рисунке \ref{fig:NO-GC1-times-heatmap} и значением на оси $Oy$ на рисунке \ref{fig:NO-GC1-times-compiled} показано отношение длительности расписания, построенного алгоритмом к длительности оптимального расписания. Значения всегда больше 1, чем меньше, тем лучше.

Точность алгоритма возрастает с увеличением количества работ и понижается с увеличением количества процессоров.

\subsubsection{Жадный алгоритм с EDF эвристикой}

\paragraph{Постановка задачи с ограничением на количество передач}

\begin{figure}[!htbp]
    \centering
    \begin{subfigure}{0.49\textwidth}
        \includegraphics[width=\textwidth]{imgs/ideal_1/CR_EDF/th.png}
        \caption{Тепловая карта}
        \label{fig:CR-EDF-times-heatmap}
    \end{subfigure}
    \hfill
    \begin{subfigure}{0.49\textwidth}
        \includegraphics[width=\textwidth]{imgs/ideal_1/CR_EDF/gr_amalgamated.png}
        \caption{Сводный график} 
        \label{fig:CR-EDF-times-compiled} 
    \end{subfigure}
    \caption{Отношение времени выполнения расписания к оптимальному времени выполнения на данных с известным оптимумом на постановке $CR$}
\end{figure}

На рисунках \ref{fig:CR-EDF-times-heatmap} и \ref{fig:CR-EDF-times-compiled} показано качество решений, генерируемых жадным алгоритмом с EDF эвристикой на данных с известным оптимумом. Цветом на рисунке \ref{fig:CR-EDF-times-heatmap} и значением на оси $Oy$ на рисунке \ref{fig:CR-EDF-times-compiled} показано отношение длительности расписания, построенного алгоритмом к длительности оптимального расписания. Значения всегда больше 1, чем меньше, тем лучше.

Точность алгоритма повышается с увеличением количества работ, однако ухудшение решения с увеличением количества процессоров в системе менее значительно, чем в жадном алгоритме с выбором по числу потомков. 

% \begin{figure}[!htbp]
%     \centering
%     \includegraphics[width=\textwidth]{imgs/ideal_1/CR_EDF/cr_3d.png}
%     \caption{12345}
% \end{figure}

\begin{figure}[!htbp]
    \centering
    \begin{subfigure}{0.49\textwidth}
        \includegraphics[width=\textwidth]{imgs/layered_class_1/CR_EDF/times.png}
        \caption{Тепловая карта}
        \label{fig:CR-layered-EDF-times-heatmap}
    \end{subfigure}
    \hfill
    \begin{subfigure}{0.49\textwidth}
        \includegraphics[width=\textwidth]{imgs/layered_class_1/CR_EDF/gr_amalgamated.png}
        \caption{Сводный график}
        \label{fig:CR-layered-EDF-times-compiled} 
    \end{subfigure}
    \caption{Отношение времени выполнения расписания, построенного при помощи жадного алгороитма с EDF эвристикой к времени выполнения расписания, построенного при помощи жадного алгоритма с выбором по числе потомков на наборе данных, основанных на слоистых данных на постановке $CR$}
\end{figure}

На рисунках \ref{fig:CR-layered-EDF-times-heatmap} и \ref{fig:CR-layered-EDF-times-compiled} показано качество решений, генерируемых жадным алгоритмом с EDF эвристикой на данных, построенных на слоистых графах. Цветом на рисунке \ref{fig:CR-layered-EDF-times-heatmap} и значением на оси $Oy$ на рисунке \ref{fig:CR-layered-EDF-times-compiled} показано отношение длительности расписании расписании, построенного жадным алгоритмом с выбором по числу потомков.

В большинстве случаев, качество решений жадного алгоритма с EDF эвристикой лучше качества решений жадного алгоритма с выбором по числу потомков, однако преимущество остается в пределах 10\%, кроме двух выбросов в районе 1000 работ. 

% \begin{figure}[!htbp]
%     \centering
%     \includegraphics[width=\textwidth]{imgs/layered_class_1/CR_EDF/cr_3d.png}
%     \caption{12345}
% \end{figure}

\begin{figure}[!htbp]
    \centering
    \begin{subfigure}{0.49\textwidth}
        \includegraphics[width=\textwidth]{imgs/unbalanced/CR_EDF/times.png}
        \caption{Тепловая карта}
        \label{fig:CR-disbalanced-EDF-times-heatmap}
    \end{subfigure}
    \hfill
    \begin{subfigure}{0.49\textwidth}
        \includegraphics[width=\textwidth]{imgs/unbalanced/CR_EDF/gr_amalgamated.png}
        \caption{Сводный график}
        \label{fig:CR-disbalanced-EDF-times-compiled} 
    \end{subfigure}
    \caption{Отношение времени выполнения расписания, построенного при помощи жадного алгороитма с EDF эвристикой к времени выполнения расписания, построенного при помощи жадного алгоритма с выбором по числе потомков на наборе данных, основанных на неоднородных процессорах на постановке $CR$}
\end{figure}

На рисунках \ref{fig:CR-disbalanced-EDF-times-heatmap} и \ref{fig:CR-disbalanced-EDF-times-compiled} показано качество решений, генерируемых жадным алгоритмом с EDF эвристикой на данных, построенных на слоистых графах. Цветом на рисунке \ref{fig:CR-disbalanced-EDF-times-heatmap} и значением на оси $Oy$ на рисунке \ref{fig:CR-disbalanced-EDF-times-compiled} показано отношение длительности расписании расписании, построенного жадным алгоритмом с выбором по числу потомков.

Алгоритм не дает значимых улучшений решения по сравнению с жадным алгоритмом с выбором по числу потомков, кроме случая с 500 работами на 64 процессорах, в котором решение, строимое жадным алгоритмом с EDF эвристикой значительно хуже решения, построенного жадным алгоритмом с выбором по числу потомков.

% \begin{figure}[!htbp]
%     \centering
%     \includegraphics[width=\textwidth]{imgs/unbalanced/CR_EDF/cr_3d.png}
%     \caption{12345}
% \end{figure}

\paragraph{Постановка задачи без ограничений}

\begin{figure}[!htbp]
    \centering
    \begin{subfigure}{0.49\textwidth}
        \includegraphics[width=\textwidth]{imgs/ideal_1/NO_EDF/th.png}
        \caption{Тепловая карта}
        \label{fig:NO-EDF-times-heatmap}
    \end{subfigure}
    \hfill
    \begin{subfigure}{0.49\textwidth}
        \includegraphics[width=\textwidth]{imgs/ideal_1/NO_EDF/gr_amalgamated.png}
        \caption{Сводный график} 
        \label{fig:NO-EDF-times-compiled}
    \end{subfigure}
    \caption{Отношение времени выполнения расписания, построенного жадным алгоритмом с EDF эвристикой к оптимальному времени выполнения расписания на постановке $NO$}
\end{figure}

На рисунках \ref{fig:NO-EDF-times-heatmap} и \ref{fig:NO-EDF-times-compiled} показано качество решений, генерируемых жадным алгоритмом с EDF эвристикой на данных с известным оптимумом. Цветом на рисунке \ref{fig:NO-EDF-times-heatmap} и значением на оси $Oy$ на рисунке \ref{fig:NO-EDF-times-compiled} показано отношение длительности расписания, построенного жадным алгоритмом с выбором по числу потомков. Значения всегда больше 1, чем меньше, тем лучше.

Алгоритм хорошо справляется с задачей, в большинстве случаев отклонение от оптимума не превышает 5\%. Для 2000 и 5000 работ точность алгоритма значительно выше таковой для жадного алгоритма с выбором по числу потомков.

\begin{figure}[!htbp]
    \centering
    \begin{subfigure}{0.49\textwidth}
        \includegraphics[width=\textwidth]{imgs/layered_class_1/NO_EDF/times.png}
        \caption{Тепловая карта}
        \label{fig:NO-layered-EDF-times-heatmap}
    \end{subfigure}
    \hfill
    \begin{subfigure}{0.49\textwidth}
        \includegraphics[width=\textwidth]{imgs/layered_class_1/NO_EDF/gr_amalgamated.png}
        \caption{Сводный график} 
        \label{fig:NO-layered-EDF-times-compiled}
    \end{subfigure}
    \caption{Отношение времени выполнения расписания, построенного жадным алгоритмом с EDF эвристикой к времени выполнения расписания, построенного жадным алгоритмом с выбором по числу потомков на данных, основанных на слоистых графах, на постановке $NO$}
\end{figure}

На рисунках \ref{fig:NO-layered-EDF-times-heatmap} и \ref{fig:NO-layered-EDF-times-compiled} показано качество решений, генерируемых жадным алгоритмом с EDF эвристикой на данных, построенных на слоистых графах. Цветом на рисунке \ref{fig:NO-layered-EDF-times-heatmap} и значением на оси $Oy$ на рисунке \ref{fig:NO-layered-EDF-times-compiled} показано отношение длительности расписания, построенного жадным алгоритмом с выбором по числу потомков.

Во всех случаях, качество решений выше качества решений жадного алгоритма с выбором по числу потомков, но это улучшение не превышает 10\%.

\begin{figure}[!htbp]
    \centering
    \begin{subfigure}{0.49\textwidth}
        \includegraphics[width=\textwidth]{imgs/unbalanced/NO_EDF/times.png}
        \caption{Тепловая карта}
        \label{fig:NO-disbalanced-EDF-times-heatmap}
    \end{subfigure}
    \hfill
    \begin{subfigure}{0.49\textwidth}
        \includegraphics[width=\textwidth]{imgs/unbalanced/NO_EDF/gr_amalgamated.png}
        \caption{Сводный график}
        \label{fig:NO-disbalanced-EDF-times-compiled}
    \end{subfigure}
    \caption{Отношение времени выполнения расписания, построенного жадным алгоритмом с EDF эвристикой к времени выполнения расписания, построенного жадным алгоритмом с выбором по числу потомков на данных, основанных на неоднородных графах, на постановке $NO$}
\end{figure}

На рисунках \ref{fig:NO-disbalanced-EDF-times-heatmap} и \ref{fig:NO-disbalanced-EDF-times-compiled} показано качество решений, генерируемых жадным алгоритмом с EDF эвристикой на данных, построенных на слоистых графах. Цветом на рисунке \ref{fig:NO-disbalanced-EDF-times-heatmap} и значением на оси $Oy$ на рисунке \ref{fig:NO-disbalanced-EDF-times-compiled} показано отношение длительности расписании расписании, построенного жадным алгоритмом с выбором по числу потомков. Значения всегда больше 1, чем меньше, тем лучше.

Во всех случаях, качество решений выше качества решений жадного алгоритма с выбором по числу потомков, но это улучшение не превышает 10\%, за исключением случая с 32 процессорами.

\subsection{Исследование временной сложности алгоритма}

\subsubsection{Жадный алгоритм с выбором по числу потомков}

\paragraph{Постановка задачи с ограничением на количество передач}

\begin{figure}[!htbp]
    \centering
    \begin{subfigure}{0.49\textwidth}
        \includegraphics[width=\textwidth]{imgs/ideal_1/CR/et_heatmap.png}
        \caption{Тепловая карта}
        \label{fig:CR-exec-time-heatmap}
    \end{subfigure}
    \hfill
    \begin{subfigure}{0.49\textwidth}
        \includegraphics[width=\textwidth]{imgs/ideal_1/CR/tr_graph.png}
        \caption{Сводный график}
        \label{fig:CR-exec-time-compiled}
    \end{subfigure}
    \caption{Время выполнения жадного алгоритма с выбором по числу потомков на данных с известным оптимумом, на постановке $CR$, в секундах}
\end{figure}

На рисунках \ref{fig:CR-exec-time-heatmap} и \ref{fig:CR-exec-time-compiled} показано время выполнения жадного алгоритма с выбором по числу потомков, включая прогоны METIS. Время выполнения растет с увеличением количества вершин. При равном количестве работ, выше время выполнения при меньшем количестве процессоров. Причина в том, что при равном количестве работ и понижении количества процессоров повышается количество работ, распределенных на процессор, что приводит к большему количеству пропусков в расписании, что значит, что алгоритм постановки работы в расписании отработает быстрее, т.к. он работает до первого найденного доступного простоя на процессоре. 

\begin{figure}[!htbp]
    \centering
    \begin{subfigure}{0.49\textwidth}
        \includegraphics[width=\textwidth]{imgs/layered_class_1/CR/et_heatmap.png}
        \caption{Тепловая карта}
        \label{fig:CR-layered-exec-time-heatmap}
    \end{subfigure}
    \hfill
    \begin{subfigure}{0.49\textwidth}
        \includegraphics[width=\textwidth]{imgs/layered_class_1/CR/tr_graph.png}
        \caption{Сводный график}
        \label{fig:CR-layered-exec-time-compiled}
    \end{subfigure}
    \caption{Время выполнения жадного алгоритма с выбором по числу потомков на данных, основанных на слоистых графах, на постановке $CR$, в миллисекундах}
\end{figure}

На рисунках \ref{fig:CR-layered-exec-time-heatmap} и \ref{fig:CR-layered-exec-time-compiled} показано время выполнения алгоритма на наборе данных, основанных на слоистых графах. 

Кроме двух выбросов, соотносящихся с самым малым количеством работ и самым большим количеством процессоров в системе, нет существенной зависимости времени выполнения от количества процессоров в системе.

\begin{figure}[!htbp]
    \centering
    \begin{subfigure}{0.49\textwidth}
        \includegraphics[width=\textwidth]{imgs/unbalanced/CR/et_heatmap.png}
        \caption{Тепловая карта}
        \label{fig:CR-disbalanced-exec-time-heatmap}
    \end{subfigure}
    \hfill
    \begin{subfigure}{0.49\textwidth}
        \includegraphics[width=\textwidth]{imgs/unbalanced/CR/tr_graph.png}
        \caption{Сводный график}
        \label{fig:CR-disbalanced-exec-time-compiled}
    \end{subfigure}
    \caption{Время выполнения жадного алгоритма с выбором по числу потомков на данных, основанных на неоднородных процессорах, на постановке $CR$, в миллисекундах}
\end{figure}

На рисунках \ref{fig:CR-disbalanced-exec-time-heatmap} и \ref{fig:CR-disbalanced-exec-time-compiled} показано время выполнения алгоритма на наборе данных, основанных на слоистых графах с неоднородными процессорами.

Время выполнения алгоритма растет с увеличением количества работ, кроме трех выбросов. Эти выбросы соответствуют самому малому количеству работ и самому высокому количеству процессоров.

\paragraph{Постановка задачи без ограничений}

\begin{figure}[!htbp]
    \centering
    \begin{subfigure}{0.49\textwidth}
        \includegraphics[width=\textwidth]{imgs/ideal_1/NO/et_heatmap.png}
        \caption{Тепловая карта}
        \label{fig:NO-exec-time-heatmap}
    \end{subfigure}
    \hfill
    \begin{subfigure}{0.49\textwidth}
        \includegraphics[width=\textwidth]{imgs/ideal_1/NO/tr_graph.png}
        \caption{Сводный график}
        \label{fig:NO-exec-time-compiled}
    \end{subfigure}
    \caption{Время выполнения жадного алгоритма с выбором по числу потомков на данных с известным оптимумом, на постановке $NO$, в секундах}
\end{figure}

На рисунках \ref{fig:NO-exec-time-heatmap} и \ref{fig:NO-exec-time-compiled} показано время выполнения алгоритма на наборе данных c известным оптимумом. Время выполнения растет с увеличением количества задач. 

При одинаковом количестве работ, быстрее выполняется алгоритм с большим количеством процессоров, по тем же причинами, что и в постановке с дополнительным ограничением $CR$. 

\begin{figure}[!htbp]
    \centering
    \begin{subfigure}{0.49\textwidth}
        \includegraphics[width=\textwidth]{imgs/layered_class_1/NO/et_heatmap.png}
        \caption{Тепловая карта}
        \label{fig:NO-layered-exec-time-heatmap}
    \end{subfigure}
    \hfill
    \begin{subfigure}{0.49\textwidth}
        \includegraphics[width=\textwidth]{imgs/layered_class_1/NO/tr_graph.png}
        \caption{Сводный график}
        \label{fig:NO-layered-exec-time-compiled}
    \end{subfigure}
    \caption{Время выполнения жадного алгоритма с выбором по числу потомков на данных, основанных на слоистых графах, на постановке $NO$, в миллисекундах}
\end{figure}

На рисунках \ref{fig:NO-exec-time-heatmap} и \ref{fig:NO-exec-time-compiled} показано время выполнения алгоритма на наборе данных, основанных на слоистых графах. Время выполнения растет с увеличением количества работ и количества процессоров.

\begin{figure}[!htbp]
    \centering
    \begin{subfigure}{0.49\textwidth}
        \includegraphics[width=\textwidth]{imgs/unbalanced/NO/et_heatmap.png}
        \caption{Тепловая карта}
        \label{fig:NO-unbalanced-exec-time-heatmap}
    \end{subfigure}
    \hfill
    \begin{subfigure}{0.49\textwidth}
        \includegraphics[width=\textwidth]{imgs/unbalanced/NO/tr_graph.png}
        \caption{Сводный график}
        \label{fig:NO-unbalanced-exec-time-compiled}
    \end{subfigure}
    \caption{Время выполнения жадного алгоритма с выбором по числу потомков на данных, основанных на неоднородных процессорах, на постановке $NO$, в миллисекундах}
\end{figure}

На рисунках \ref{fig:NO-unbalanced-exec-time-heatmap} и \ref{fig:NO-unbalanced-exec-time-compiled} показано время выполнения алгоритма на наборе данных c неоднородными процессорами. Время выполнения растет с увеличением количества работ и количества процессоров.

\subsubsection{Жадный алгоритм с EDF эвристикой}

\paragraph{Постановка задачи с ограничением на количество передач}

\begin{figure}[!htbp]
    \centering
    \begin{subfigure}{0.49\textwidth}
        \includegraphics[width=\textwidth]{imgs/ideal_1/CR_EDF/et_heatmap.png}
        \caption{Тепловая карта}
        \label{fig:CR-EDF-exec-time-heatmap}
    \end{subfigure}
    \hfill
    \begin{subfigure}{0.49\textwidth}
        \includegraphics[width=\textwidth]{imgs/ideal_1/CR_EDF/tr_graph.png}
        \caption{Сводный график}
        \label{fig:CR-EDF-exec-time-compiled}
    \end{subfigure}
    \caption{Время выполнения жадного алгоритма с EDF эвристикой на данных с известным оптимумом, на постановке $CR$, в секундах}
\end{figure}

На рисунках \ref{fig:CR-EDF-exec-time-heatmap} и \ref{fig:CR-EDF-exec-time-compiled} показано время выполнения жадного алгоритма с EDF эвристикой, включая прогоны METIS. Время выполнения растет с увеличением количества вершин, при этом в несколько раз меньшим времени, затраченного на прогон жадного алгоритма с выбором по числу потомков. При равном количестве работ, выше время выполнения при меньшем количестве процессоров. Причина схожа с причинами подобного явления для жадного алгоритма с выбором по числу потомков, поскольку они разделяют одну процедуру поиска нового места в расписании. 

\begin{figure}[!htbp]
    \centering
    \begin{subfigure}{0.49\textwidth}
        \includegraphics[width=\textwidth]{imgs/layered_class_1/CR_EDF/et_heatmap.png}
        \caption{Тепловая карта}
        \label{fig:CR-layered-EDF-exec-time-heatmap}
    \end{subfigure}
    \hfill
    \begin{subfigure}{0.49\textwidth}
        \includegraphics[width=\textwidth]{imgs/layered_class_1/CR_EDF/tr_graph.png}
        \caption{Сводный график}
        \label{fig:CR-layered-EDF-exec-time-compiled}
    \end{subfigure}
    \caption{Время выполнения жадного алгоритма с EDF эвристикой на данных, основанных на слоистых данных, на постановке $CR$, в миллисекундах}
\end{figure}

На рисунках \ref{fig:CR-layered-EDF-exec-time-heatmap} и \ref{fig:CR-layered-EDF-exec-time-compiled} показано время выполнения жадного алгоритма с EDF эвристикой, включая прогоны METIS, на слоистых данных, основанных на слоистых графах. Как и для жадного алгоритма с выбором по числу потомков, на данных видно два выброса, которые соотносятся с самым большим количеством процессором и самым маленьким количеством работ в исходных данных. Также не существует значимой зависимости между количеством процессоров в системе и временем, затраченным на построение расписания. Алгоритм работает в несколько раз быстрее жадного алгоритма с выбором по числу потомков.

\begin{figure}[!htbp]
    \centering
    \begin{subfigure}{0.49\textwidth}
        \includegraphics[width=\textwidth]{imgs/unbalanced/CR_EDF/et_heatmap.png}
        \caption{Тепловая карта}
        \label{fig:CR-unbalanced-EDF-exec-time-heatmap}
    \end{subfigure}
    \hfill
    \begin{subfigure}{0.49\textwidth}
        \includegraphics[width=\textwidth]{imgs/unbalanced/CR_EDF/tr_graph.png}
        \caption{Сводный график}
        \label{fig:CR-unbalanced-EDF-exec-time-compiled}
    \end{subfigure}
    \caption{Время выполнения жадного алгоритма с EDF эвристикой на данных, основанных на неоднородных процессорах, на постановке $CR$, в миллисекундах}
\end{figure}

На рисунках \ref{fig:CR-unbalanced-EDF-exec-time-heatmap} и \ref{fig:CR-unbalanced-EDF-exec-time-compiled} показано время выполнения жадного алгоритма с EDF эвристикой, включая прогоны METIS, на слоистых данных с неоднородными процессорами. Как и для жадного алгоритма с выбором по числу потомков, на данных видно три выброса, которые соотносятся с самым большим количеством процессором и самым маленьким количеством работ в исходных данных. Также не существует значимой зависимости между количеством процессоров в системе и временем, затраченным на построение расписания. Алгоритм работает в несколько раз быстрее жадного алгоритма с выбором по числу потомков.

\paragraph{Постановка задачи без ограничений}

\begin{figure}[!htbp]
    \centering
    \begin{subfigure}{0.49\textwidth}
        \includegraphics[width=\textwidth]{imgs/ideal_1/NO_EDF/et_heatmap.png}
        \caption{Тепловая карта}
        \label{fig:NO-EDF-exec-time-heatmap}
    \end{subfigure}
    \hfill
    \begin{subfigure}{0.49\textwidth}
        \includegraphics[width=\textwidth]{imgs/ideal_1/NO_EDF/tr_graph.png}
        \caption{Сводный график}
        \label{fig:NO-EDF-exec-time-compiled}
    \end{subfigure}
    \caption{Время выполнения жадного алгоритма с EDF эвристикой на данных с известным оптимумом, на постановке $NO$, в секундах}
\end{figure}

На рисунках \ref{fig:NO-EDF-exec-time-heatmap} и \ref{fig:NO-EDF-exec-time-compiled} показано время выполнения жадного алгоритма с EDF эвристикой, включая прогоны METIS, на данных с известным оптимумом.

\begin{figure}[!htbp]
    \centering
    \begin{subfigure}{0.49\textwidth}
        \includegraphics[width=\textwidth]{imgs/layered_class_1/NO_EDF/et_heatmap.png}
        \caption{Тепловая карта}
        \label{fig:NO-layered-EDF-exec-time-heatmap}
    \end{subfigure}
    \hfill
    \begin{subfigure}{0.49\textwidth}
        \includegraphics[width=\textwidth]{imgs/layered_class_1/NO_EDF/tr_graph.png}
        \caption{Сводный график}
        \label{fig:NO-layered-EDF-exec-time-compiled}
    \end{subfigure}
    \caption{Время выполнения жадного алгоритма с EDF эвристикой на данных, основанных на слоистых данных, на постановке $NO$, в миллисекундах}
\end{figure}

На рисунках \ref{fig:NO-layered-EDF-exec-time-heatmap} и \ref{fig:NO-layered-EDF-exec-time-compiled} показано время выполнения жадного алгоритма с EDF эвристикой, включая прогоны METIS, на данных, основанных на слоистых графах.

\begin{figure}[!htbp]
    \centering
    \begin{subfigure}{0.49\textwidth}
        \includegraphics[width=\textwidth]{imgs/unbalanced/NO_EDF/et_heatmap.png}
        \caption{Тепловая карта}
        \label{fig:NO-unbalanced-EDF-exec-time-heatmap}
    \end{subfigure}
    \hfill
    \begin{subfigure}{0.49\textwidth}
        \includegraphics[width=\textwidth]{imgs/unbalanced/NO_EDF/tr_graph.png}
        \caption{Сводный график}
        \label{fig:NO-unbalanced-EDF-exec-time-compiled}
    \end{subfigure}
    \caption{Время выполнения жадного алгоритма с EDF эвристикой на данных, основанных на неоднородных процессорах, на постановке $NO$, в миллисекундах}
\end{figure}

На рисунках \ref{fig:NO-unbalanced-EDF-exec-time-heatmap} и \ref{fig:NO-unbalanced-EDF-exec-time-compiled} показано время выполнения жадного алгоритма с EDF эвристикой, включая прогоны METIS, на слоистых данных с неоднородными процессорами. Алгоритм выполняется быстрее жадного алгоритма с выбором по числу потомков, однако разница во времени выполнения незначительна. Время выполнения алгоритма увеличивается с увеличением количества работ и процессоров в системе.

\subsection{Выводы из экспериментального исследования}

На данных с известным оптимумом, для постановки с дополнительным ограничением $CR$ жадный алгоритм с выбором по числу потомков достигает точности в 40\%, а жадный алгоритм с EDF эвристикой - 5-10\%. Однако, такое превосходство не наблюдается на наборах данных, основанных на слоистых данных и данных с неоднородными процессорами, где преимущество жадного алгоритма с EDF эвристикой над жадным алгоритмом с выбором по числу потомков не превышает 10-15\%. 

При постановке без дополнительных ограничений оба алгоритма быстро достигают точности в 5\% на данных с известным оптимумом, а на данных, основанных на слоистых графах и данных, основанных на неоднородных процессорах жадный алгоритм с EDF эвристикой имеет преимущество над жадным алгоритмом с выбором по числу потомков, достигающее 15-20\%.

Предпочтительным является жадный алгоритм с EDF эвристиками, т.к. при схожей или превосходящей точности он выполняется быстрее.