Код реализации выложен в репозитории \cite{Repository}
В программной реализации были использованы следующие библиотеки:
\begin{enumerate}
    \item \mintinline{bash}{boost} 1.81 \cite{boost_framework}
    \item \mintinline{bash}{json} 3.11.2 \cite{json_lib}
    \item \mintinline{bash}{toml11} 3.7.1 \cite{toml11_lib}
\end{enumerate}

Проект обладает следующей структурой:
\begin{enumerate}
    \item \mintinline{bash}{greedy/greedy_algo.cpp}
    \item \mintinline{bash}{greedy/schedule.cpp}
    \item \mintinline{bash}{greedy/time_schedule.cpp}
    \item \mintinline{bash}{graph_part.cpp}
    \item \mintinline{bash}{input_class.cpp}
    \item \mintinline{bash}{json_dumper.cpp}
    \item \mintinline{bash}{logger_config.cpp}
    \item \mintinline{bash}{options.cpp}
    \item \mintinline{bash}{parser.cpp}
\end{enumerate}

Для сборки проекта используется \mintinline{bash}{CMake}.
\begin{minted}{bash}
    mkdir build
    cd build
    cmake ..
    make
\end{minted}

Для сборки документации (на английском) используется \mintinline{bash}{Doxygen}.
\begin{minted}{bash}
    doxygen Doxyfile
\end{minted}