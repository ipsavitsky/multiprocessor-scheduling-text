% Классическая задача построения расписания хорошо изучена и досконально описана в \cite{Coffman}. Поскольку данная задача принадлежит к классу NP-трудных, неизвестен алгоритм, который за полиномиальное время даст точный ответ, но существуют алгоритмы, которые дают приближенные результаты. Большинство таких алгоритмов разделяются на два класса: \textit{детерминированные} и \textit{недетерминированные}. Данная работа рассматривает только детерминированные алгоритмы.
% \begin{enumerate}
%     \item Алгоритмы, основанные на поиске максимального потока в сети
%     \item Алгоритмы, основанные на методах динамического программирования
%     \item Алгоритмы, основанные на методе ветвей и границ
%     \item Жадные алгоритмы
% \end{enumerate}
Общая задача построения списочных многопроцессорных расписаний заключается в следующем \cite{Shakhbazyan_1981,Coffman}. Задан ориентированный ациклический граф потока данных. Вершинами графа являются работы. Для каждой работы задано время, требуемое для ее выполнения. Дуги задают отношение частичного порядка на множестве работ. Если от $i$-ой вершины идет дуга к $j$-ой вершине, то работа $j$ может начать выполняться только после завершения работы $i$. Требуется построить расписание выполнения работ на заданном числе процессоров. Расписание корректно если выполняются следующие ограничения:
\begin{enumerate}
    \item каждая работа должна быть назначена на процессор;
    \item любую работу обслуживает лишь один процессор;
    \item частичный порядок, заданный в графе потока данных, сохранен в расписании.
\end{enumerate}

Целевой функцией является время выполнения расписания. Прерывания работ недопустимы.

Рассматриваемая в работе задача построения списочных расписаний отличается от общей задачи наличием дополнительного ограничения на корректность расписаний. Задается ограничение на максимально возможное число передач между процессорами. Для решения задач построения расписаний большой размерности на практике обычно используются жадные алгоритмы \cite{Cormen}. Их достоинством является низкая вычислительная сложность. Основным недостатком жадных алгоритмов является сильная зависимость качества получаемых решений от класса исходных данных.

Наличие дополнительного ограничения может приводить к тому, что жадные алгоритмы могут заходить в тупик: есть еще не распределенные в расписание работы, но ни одна из них не может быть размещена в расписание без нарушения дополнительных ограничений на корректность расписания. Для того чтобы исправить этот недостаток был предложен алгоритм, основанный на подходе сочетающий жадные стратегии и ограниченный перебор \cite{Kostenko_2017}. Однако при небольшой глубине перебора как показало экспериментальное исследование он также может заходить в тупик. Гарантировано он не будет заходить в тупик если глубина перебора равна числу работ, но при этом он будет иметь не полиномиальную сложность.

В работе рассматриваются два алгоритма построения списочных расписаний для классической задачи и с дополнительным ограничением на количество межпроцессорных передач, сочетающих жадные стратегии и алгоритм разбиения графа потока данных на подграфы. 
% Алгоритм разбиения графов на заданное число подграфов, минимизирует число ребер между полученными подграфами. Если подать на вход алгоритму граф потока данных, выбрать число подграфов равное числу процессоров и сопоставить каждый выходной подграф процессору, то получится распределение задач по процессорам с минимизацией числа межпроцессорных передач. Последовательность выполнения работ на процессорах определяется жадным алгоритмом.