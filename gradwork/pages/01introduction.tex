Классическая задача построения расписания хорошо изучена и досканально описана в \cite{Coffman}. Поскольку данная задача принадлежит к классу NP-трудных, неизвестен алгоритм, который за полиномиальное время даст точный ответ, но существуют алгоритмы, которые дают приближенные результаты. Большинство таких алгоритмов резделяются на два класса: \textit{детерминированные} и \textit{недетерминированные}. Данная работа рассматривает только детерминированные алгоритмы.
\begin{enumerate}
    \item Алгоритмы, основанные на поиске максимального потока в сети
    \item Алгоритмы, основанные на методах динамического программирования
    \item Алгоритмы, основанные на методе ветвей и границ
    \item Жадные алгоритмы
\end{enumerate}
