В данных, присланных от Хуавей существует разделение на 2 класса.
\begin{enumerate}
    \item Первый класс (примеры DAG\_A и DAG\_B) характеризуется относительно небольшим масштабом графа работ, небольшим числом процессоров, полнотой графа связности процессоров и одинаковыми задержками между любыми двумя процессорами.
    \item Второй класс (примеры DAG\_C и DAG\_D) характеризуется относительно большим масштабом графа работ, большим числом процессоров
\end{enumerate}
\begin{table}[!htbp]
    \begin{tabular}{c|c|c|c|c}
        & \multicolumn{4}{c}{Примеры входных данных}                            \\
        \hline
        Критерии            & DAG\_A                                     & DAG\_B & DAG\_C & DAG\_D \\
        \hline
        Масштаб графа работ & \makecell{45 вершин;                                                  \\75 ребер}                        & \makecell{1121 вершина;\\6229 ребер} & \makecell{197494 вершин;\\719389 ребер} & \makecell{1823309 вершин;\\6172920 ребер} \\
        \hline
        \makecell{Разброс                                                                           \\длительностей работ}            & 1-10                                       & 1-10   & \makecell{все работы\\одной длины} & \makecell{все работы\\одной длины} \\
        \hline
        \makecell{Связность                                                                         \\графа работ}                  & 1.66                                       & 5.55   & 3.64                   & 3.83                   \\
        \hline
        \makecell{Количество                                                                        \\процессоров}                 & 2                                          & 10     & 256                    & 4096                   \\
        \hline
        \makecell{Полный граф                                                                       \\связности\\процессоров}      & да                                         & да     & нет                    & да                     \\
        \hline
        \makecell{Одинаковые задержки                                                               \\на передачу данных} & да                                         & да     & да                     & нет                    \\
    \end{tabular}
    \caption{Сравнение примеров из классов данных}
\end{table}
