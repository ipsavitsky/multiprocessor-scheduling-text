% Обзоры других алгоритмов приведены в \cite{Shakhbazyan_1981,Davis_2011}
\subsection{Критерии обзора}
Ниже приведены критерии, по которым будут рассматриваться и сравниваться алгоритмы
\begin{enumerate}
    \item Насколько сильно рассматриваемая в статье задача отличается от решаемой. Можно ли взять алгоритм, описанный в статье за базу для алгоритма для решения данной задачи.
    \item Порядок сложности алгоритма.
    \item На данных какой размерности протестирован алгоритм.
\end{enumerate}
\subsection{Конструктивные алгоритмы}
\subsubsection{Жадные алгоритмы}
Жадные алгоритмы подразумевают декомпозицию задачи на ряд более простых подзадач. На каждом шаге решение принимается исходя из принципа получения оптимального решения для очередной подзадачи. То есть, на каждом шаге алгоритм делает выбор, оптимальный с точки зрения получения решения очередной подзадачи, предполагая, что эти локально-оптимальные решения приведут к приемлемому решению задачи. Какие-либо жадные стратегии, гарантированно получающие оптимальное расписание, на настоящий момент времени неизвестны, за исключением небольшого числа вариантов задач составления расписаний не принадлежащих к классу NP-полных. Например, известен жадный алгоритм, получающий точное решение для задачи обслуживания одним процессором максимального числа работ из заданного набора работ с фиксированными сроками начала и окончания \cite{Cormen}. Набор локальных критериев оптимизации сильно зависит от класса архитектуры. Для архитектур, в которых возможно последействие (распределяемый в расписание рабочий интервал оказывает влияние на времена инициализации ранее распределенных рабочих интервалов) возникает проблема выбора локальных критериев оптимизации, позволяющих учесть эффект последействия (на настоящий момент времени какие-либо обоснованные решения этой проблемы не известны). Кроме того, единого локального критерия (или набора и способа их использования), приводящего к наилучшему конечному результату, для решения всех подзадач не существует. Более того, при усложнении архитектуры набор и способ использования локальных критериев оказывает более сильное влияние на конечный результат. Таким образом, применение жадных алгоритмов для составления расписаний классом архитектур без последействия или даже без разделяемых ресурсов, если их влияние на значение функции построения временной диаграммы не может быть локализовано, а также проблемой выбора критериев оптимизации индивидуально для каждой подзадачи.

\subsubsection{Структкра жадных алгоритмов построения многопроцессорного расписания.}
При построении расписания жадным алгоритмом для каждой заадчи необходимо определить два параметра:
\begin{enumerate}
    \item Привязку задачи к процессору $p_i$
    \item Время старта задачи на процессоре $s_i$
\end{enumerate}
Каждый из этих параметров может быть определен своим жадным критерием. Время старта (как показано в \cite{Kalashnikov_2004}) единственным образом определяется из порядка работ на процессоре. Следовательно, имеет роль порядок, в котором работы добавляются в расписание.

Таким образом, для построения расписания достаточно определить:
\begin{enumerate}
    \item Привязку задачи к процессору $p_i$
    \item Ее номер в очереди на добавление в расписание $q_i$
\end{enumerate}

\subsubsection{Другие конструктивные алгоритмы}
(Написать про \cite{heterog_review})

\subsection{Итерационные алгоритмы}

Итерационные алгоритмы работают путем создания аппроксимированного варианта решения и последующего его улучшения. Однако, большинство из них рандомизированные и, следовательно, из нескольких различных запусков теоретически возможно получить различные расписания (несмотря на то то они все сходятся). Более того, многие из таких алгоритмов плохо масштабируемы. Муравьиные алгоритмы разобраны в \cite{Shtovba_2005}, имитации отжига - в \cite{Kirkpatrick_1983}.
\begin{table}[H]
    \caption{Существующие алгоритмы}
    \begin{tabular}{ c | c | c | c  }
        \makecell{Название алгоритма}       & Рандомизированность & Итерационный   & \makecell{Возможность \\ масштабирования} \\
        \hline
        \makecell{Генетические алгоритмы}   & Рандомный           & Итерационный   & +/-                   \\
        \makecell{Алгоритм имитации отжига} & Рандомный           & Итерационный   & +                     \\
        \makecell{Муравьиные алгоритмы}     & Рандомный           & Итерационный   & -                     \\
        \makecell{Жадные стратегии}         & Детерминированный   & Конструктивный & +                     \\
    \end{tabular}

\end{table}

Обзоры этих алгоритмов для схожих задач представлены в \cite{Coffman,Davis_2011,Shakhbazyan_1981}

\begin{itemize}
    \item Достоинства:
          \begin{enumerate}
              \item Поскольку большинство из итерационных методов - рандомизированные алгоритмы, они меньше зависят от подбора параметров.
              \item Некоторые из итерационных алгоритмов, например, муравьиные алгоритмы, могут адаптироваться к изменению начальных условий, что не релевантно в рассматриваемой постановке задачи.
          \end{enumerate}
    \item Недостатки:
          \begin{enumerate}
              \item Некоторые из итерационных алгоритмов могут быть плохо масштабируемы.
              \item Многие из итерационных алгоритмов требуют генерации начальных состояний расписаний, от которых они могут сильно зависеть.
          \end{enumerate}
\end{itemize}