% Обзоры других алгоритмов приведены в \cite{Shakhbazyan_1981,Davis_2011}
\subsection{Критерии обзора}
Ниже приведены критерии, по которым будут рассматриваться и сравниваться алгоритмы
\begin{enumerate}
    \item Насколько сильно рассматриваемая в статье задача отличается от решаемой. Можно ли взять алгоритм, описанный в статье за базу для алгоритма для решения данной задачи.
    \item Точность решений, строимых рассмотренными алгоритмов
    \item Порядок сложности алгоритма, насколько он масштабируемый.
    \item На данных какой размерности протестирован алгоритм. Время его работы.
\end{enumerate}
\subsection{Конструктивные алгоритмы}
\subsubsection{Жадные алгоритмы}
Жадные алгоритмы подразумевают декомпозицию задачи на ряд более простых подзадач. На каждом шаге решение принимается исходя из принципа получения оптимального решения для очередной подзадачи. То есть, на каждом шаге алгоритм делает выбор, оптимальный с точки зрения получения решения очередной подзадачи, предполагая, что эти локально-оптимальные решения приведут к приемлемому решению задачи. Какие-либо жадные стратегии, гарантированно получающие оптимальное расписание, на настоящий момент времени неизвестны, за исключением небольшого числа вариантов задач составления расписаний не принадлежащих к классу NP-полных. Например, известен жадный алгоритм, получающий точное решение для задачи обслуживания одним процессором максимального числа работ из заданного набора работ с фиксированными сроками начала и окончания \cite{Cormen}. Набор локальных критериев оптимизации сильно зависит от класса архитектуры. Для архитектур, в которых возможно последействие (распределяемый в расписание рабочий интервал оказывает влияние на времена инициализации ранее распределенных рабочих интервалов) возникает проблема выбора локальных критериев оптимизации, позволяющих учесть эффект последействия (на настоящий момент времени какие-либо обоснованные решения этой проблемы не известны). Кроме того, единого локального критерия (или набора и способа их использования), приводящего к наилучшему конечному результату, для решения всех подзадач не существует. Более того, при усложнении архитектуры набор и способ использования локальных критериев оказывает более сильное влияние на конечный результат. Таким образом, применение жадных алгоритмов для составления расписаний классом архитектур без последействия или даже без разделяемых ресурсов, если их влияние на значение функции построения временной диаграммы не может быть локализовано, а также проблемой выбора критериев оптимизации индивидуально для каждой подзадачи.

\subsubsection{Структкра жадных алгоритмов построения многопроцессорного расписания.}
При построении расписания жадным алгоритмом для каждой заадчи необходимо определить два параметра:
\begin{enumerate}
    \item Привязку задачи к процессору $p_i$
    \item Время старта задачи на процессоре $s_i$
\end{enumerate}
Каждый из этих параметров может быть определен своим жадным критерием. Время старта (как показано в \cite{Kalashnikov_2004}) единственным образом определяется из порядка работ на процессоре. Следовательно, имеет роль порядок, в котором работы добавляются в расписание.

Таким образом, для построения расписания достаточно определить:
\begin{enumerate}
    \item Привязку задачи к процессору $p_i$
    \item Ее номер в очереди на добавление в расписание $q_i$
\end{enumerate}

\subsubsection{Алгоритмы, основанные на методе динамического программирования}

Алгоритмы динамического программирования разбивают сложную задачу на более простые подзадачи и находят обратную связь между их оптимальными решениями. Этот метод позволяет более эффективно решить основную проблему. Алгоритмы из этой области могут предоставлять глобальные оптимальные решения, но их недостатком является неполиномиальная сложность. В частности, с точки зрения NP-сложных задач планирования сложность этих алгоритмов является экспоненциальной функцией размера входных данных. Кроме того, для нахождения обратной зависимости между оптимальными решениями и подзадачами необходима модель аналитической системы. Это означает, что алгоритм невыполним, так как имеет высокий уровень сложности.

\subsubsection{Алгоритмы, основанные на методе ветвей и границ}

Алгоритм ветвей и границ является эффективным методом решения производных задач оптимизации. Этот метод делит пространство потенциальных решений на различные области, дает точные оценки их значения по отношению к целевой функции и сокращает ненужные участки, где невозможно найти оптимальное решение. Хотя глобальные оптимальные решения могут быть получены с помощью алгоритмов, сложность этих алгоритмов для большинства задач комбинаторной оптимизации неполиномиальна. Особенно для NP-сложных задач планирования сложность является факториальной функцией размера входных данных. Следовательно, эти алгоритмы неадекватны для решения проблемы.

Алгоритм, обсуждаемый в [M. Rahman, “Branch and Bound Algorithm for Multiprocessor Scheduling,” M.S. thesis, Dept. Comput. Eng., Dalarna Univ., Sweden, 2009], был протестирован с использованием наборов данных с числом процессоров до 16 и графов, содержащих до 100 вершин. Результаты показали ожидаемый экспоненциальный рост времени работы.

\subsubsection{Алгоритмы, основанные на нахождении максимального потока в сети}

Алгоритмические методы составления многопроцессорных расписаний включают поиск максимального потока в транспортной сети. Этот метод, по сути, переводит процесс построения расписания в поиск максимально возможного потока в указанной сети. Указанная сеть строится на основе набора задач, процессоров и параметров исходного задания. После построения сети выполняется процесс поиска максимального потока. Затем, с помощью значений потока в сети, может быть достигнуто многопроцессорное планирование. К сожалению, этот конкретный алгоритм подходит только для задач, допускающих какое-либо прерывание. Поэтому в текущей ситуации его нельзя использовать.

\subsubsection{Другие конструктивные алгоритмы}
(Написать про \cite{heterog_review})

\subsection{Итеративные алгоритмы}

\subsubsection{Генетические алгоритмы}

Алгоритмы, использующие механизмы естественной эволюции, такие как генетические и эволюционные алгоритмы, очень эффективны. Три основных фактора, влияющих на процесс эволюции, представляют собой их взаимодействие и приводят к решению многих проблем. Вариации являются источником новых физических и поведенческих характеристик живых организмов. Наследственные факторы обеспечивают неизменность этих черт. Естественный отбор уничтожает организмы, которые не справляются с условиями окружающей среды.

Недостатком этого типа алгоритмической технологии является отсутствие масштабируемости. По мере увеличения использования становится все труднее управлять системой или масштабировать ее для различных задач. Результаты исследования в статье демонстрируют, что алгоритм не имеет высокой производительности при использовании больших объемов данных. По мере увеличения размера графов также увеличиваются операции и поколения, что сильно влияет на его масштабируемость. Этот алгоритм, который оказался самым быстрым среди всех, смог обработать задачу, содержащую 1000 заданий, за 1,5 часа на процессоре с частотой 2 ГГц. Хотя это может показаться длительным временем выполнения, это в значительной степени связано с тем, что генетические и эволюционные алгоритмы требуют наличия множества решений, доступных в популяциях для реализации. Один из основных недостатков этого алгоритма вызывает беспокойство.

В предыдущих исследованиях доля межпроцессорных передач не учитывалась. Здесь мы уделяем больше внимания времени передачи данных между заданиями, а не продолжительности переключения между процессорами. Это изменение того, что первоначально рассматривалось по этому вопросу. Чтобы сделать алгоритм пригодным для решения определенной задачи, операции мутации и кроссовера необходимо скорректировать с учетом границ межпроцессорной передачи данных. Это необходимо для того, чтобы применить алгоритм.

\subsubsection{Алгоритм имитации отжига}

Алгоритм имитации отжига, представленный в 1983 году, — отличный способ найти приближенные решения для сложных NP-сложных задач комбинаторной оптимизации. Это позволяет вам решить эти проблемы, используя ряд стохастических шагов, гарантируя, что вы получите наилучшее возможное решение. Чтобы понять, как этот процесс работает на практике, ознакомьтесь со статьей о принципах его работы.

Связанная с этим проблема была изучена ранее, которая сосредоточена на минимизации времени выполнения расписания. Входными данными здесь был график прямой зависимости заданий с известными вычислительными сложностями от процессоров системы.

При отсутствии затрат на передачу данных необходимо скорректировать целевую функцию с учетом этого. Это можно сделать, включив расчет времени передачи данных в алгоритм при построении временной диаграммы. Помимо вышеупомянутого различия, ограничения межпроцессорной передачи и балансировки не учитываются. Чтобы устранить это ограничение, необходимо исправить функцию, используемую для оценки качества результатов.

Симулированный отжиг может эффективно обрабатывать большие наборы данных, поскольку он работает с одним решением. В отличие от генетических и эволюционных алгоритмов, это делает его более масштабируемым.

\subsubsection{Алгоритм муравьиных колоний}

В процессе поиска оптимального пути, муравьиные алгоритмы используют два механизма: положительную обратную связь (чем больше муравьев проходит по определенному пути, тем больше феромонов они оставляют, что привлекает еще больше муравьев) и отрицательную обратную связь (феромоны испаряются со временем, что означает, что меньше муравьев будет выбирать этот путь, если он не является оптимальным).

Муравьиные алгоритмы широко применяются в задачах оптимизации, таких как поиск кратчайшего пути в графах, распределение ресурсов и т.д. Они также могут использоваться в сочетании с другими методами оптимизации для достижения еще более точных результатов.

Кроме того, муравьиные алгоритмы могут столкнуться с проблемой преждевременной сходимости, когда муравьи начинают следовать по одному и тому же пути, не исследуя другие возможности. Это может привести к тому, что оптимальное решение не будет найдено.

Также стоит отметить, что муравьиные алгоритмы не всегда являются лучшим выбором для решения задач оптимизации. Например, они могут быть менее эффективными в случае больших наборов данных или когда имеется много локальных оптимумов.

В целом, муравьиные алгоритмы представляют собой интересный подход к решению задач оптимизации, основанный на принципах самоорганизации и эмерджентности. Они могут быть эффективными в некоторых случаях, но требуют тщательной настройки параметров и учета особенностей конкретной задачи.

% Итерационные алгоритмы работают путем создания аппроксимированного варианта решения и последующего его улучшения. Однако, большинство из них рандомизированные и, следовательно, из нескольких различных запусков теоретически возможно получить различные расписания (несмотря на то то они все сходятся). Более того, многие из таких алгоритмов плохо масштабируемы. Муравьиные алгоритмы разобраны в \cite{Shtovba_2005}, имитации отжига - в \cite{Kirkpatrick_1983}.
% \begin{table}[H]
%     \begin{tabularx}{\textwidth}{ X | l | X | X  }
%         Название алгоритма       & Рандомизированность & Итерационный   & Возможность  масштабирования \\
%         \hline
%         Генетические алгоритмы   & Рандомный           & Итерационный   & +/-                          \\
%         Алгоритм имитации отжига & Рандомный           & Итерационный   & +                            \\
%         Муравьиные алгоритмы     & Рандомный           & Итерационный   & -                            \\
%         Жадные стратегии         & Детерминированный   & Конструктивный & +                            \\
%     \end{tabularx}
%     \caption{Существующие алгоритмы}
%     \label{tbl:review}
% \end{table}

% Обзоры этих алгоритмов для схожих задач представлены в \cite{Coffman,Davis_2011,Shakhbazyan_1981}
