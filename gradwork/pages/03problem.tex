В задаче построения расписания заданы:
\begin{enumerate}
    \item Граф потока управления $G$ без циклов, в котором дуги - зависимости по данным, а вершины - задания. Вершин $n$, дуг $m$
    \item Вычислительная система, состоящая из $p$ процессоров.
    \item Матрица $C_{n \times p}$ времени выполнения работ на процессорах. Каждая строка этой матрицы - длины выполнения $n$-й задачи на $p$ процессорах.
          % \item Матрица $D_{kl}$ передач данных между процессорами, $k=1 \dots p, l = 1 \dots p, D_{kk} = 0$. $D_{ij}$-й элемент этой матрицы - время передачи данных между процессорами $i$ и $j$.
    \item Время $d$, затрачиваемое на межпроцессорную передачу.
\end{enumerate}

Расписание определено, если заданы:
\begin{enumerate}
    \item Множества процессоров и работ
    \item Привязка - всюду определенная на множестве работ функция, которая задает распределение работ по процессорам
    \item Порядок - заданные ограничения на последовательность выполнения работ и являющиеся отношением частичного порядка, удовлетворяющее условиям ацикличности и транзитивности. Отношение порядка на множестве работ, распределенных на один процессор, является отношением полного порядка.
\end{enumerate}

\begin{figure}[!htbp]
    \ctikzfig{schedule-graph}
    \caption{Граф $G$ потока данных}
    \label{fig:DFG}
\end{figure}
Пусть дан следующий граф потока данных, изображенный на рисунке \ref{fig:DFG}.

Пусть в оптимальном расписании работы $3$ и $9$ будут поставлены на процессор $Pr1$. $2$, $5$, $8$ - на $Pr2$, а $1$, $10$, $4$ и $6$ - на $Pr3$. Рассмотрим как такое расписание будет выглядеть в различных представлениях:
\begin{enumerate}
    \item Графическое представление.

          \begin{figure}[!htbp]
              \ctikzfig{schedule-graphical-form}
              \caption{Графическое представление расписания}
              \label{fig:graphical-form}
          \end{figure}
          В такой форме представления расписания, изображенной на рисунке 2, каждой задаче сопоставляется пара из номера процессора и порядкового номера работы на процессоре, а так же секущие дуги, если задачи зависят друг от друга.
    \item Временная диаграмма.

          \begin{figure}[!htbp]
              \ctikzfig{schedule-time-diagram-new}
              \caption{Представление расписания в виде временной диаграммы}
              \label{fig:time-diagram}
          \end{figure}

          % \begin{figure}[!htbp]
          %     \ctikzfig{schedule-time-diagram}
          % \end{figure}

          В такой форме представления, показанной на рисунке \ref{fig:time-diagram}, расписания каждой задаче сопоставляется пара из номера процессора и времени старта задачи на процессоре.
\end{enumerate}
Доказано, что эти представления полностью эквивалентны и, имея одно, возможно построить другое. В предложенном решении расписание строится в виде временной диаграммы. Стоит отметить, что во временной диаграмме иогут возникать простои в расписании, как, например, на рисунке \ref{fig:time-diagram} в системе с длиной межпроцессорной передачи $d=1$, между работами $T_3$ и $T_9$ на процессоре $Pr1$, поскольку работа $T_9$ зависит от работы $T_5$, которая размещена на другой процессор.

Из входных данных задачи требуется построить расписание $HP$, то есть для $i$-й работы определить время начала ее выполнения $s_i$ и процессор $p_i$ на котором она будет выполняться, минимизировать время выполнения расписания, выполнив дополнительные ограничения.


Ограничения на корректность расписания следующие:
\begin{enumerate}
    \item Каждый процессор может одновременно выполнять не больше одной работы;
    \item Прерывание работ недопустимо, перенос частично выполненной работы на другой процессор недопустим;
    \item Если между двумя работами есть зависимость на данным, то между завершением работы-отправителя и стартом работы-получателя должен быть интервал времени, не меньший, чем задержка на передачу данных между ними (с учетом привязки работ к процессорам и маршрута передачи данных).
\end{enumerate}
\subsection{Различные постановки задачи} \label{sec:crit}
\begin{enumerate}
    \item Задача без дополнительных ограничений на расписание. Далее эта постановка будет упоминаться как \textbf{постановка $NO$}
    \item Задача с дополнительным ограничением на количество передач:
          \begin{itemize}
              \item $CR = \frac{m_{ip}}{m} < 0.4$, где $m_{ip}$ - количество передач данных между работами на каждый процессор. Далее эта постановка будет упоминаться как \textbf{постановка $CR$}
          \end{itemize}
\end{enumerate}