\subsection{Дополнительные обозначения}
\begin{enumerate}
    \item $D= \left( d_1, d_2, \dots, d_l \right)$, где $l$ - количество вершин, доступных для добавления(т.е. у которых нет предшественников в графе $G$) - множество вершин, доступных для добавления в расписание.
    \item $\left( s_i, p_i \right)$ - пара, состоящая из номера процессора $p_i$ и время старта задачи $s_i$, то есть достаточное количество информации для размещения работы в расписании.
\end{enumerate}
\subsubsection{Жадные критерии}
\begin{enumerate}
    \item $GC1$ - критерий, используемый в выборе работы на постановку
    \item $GC2$ - критерий, используемый в выборе места постановки работы в расписание
\end{enumerate}

\subsection{Жадный алгоритм}
\begin{enumerate}
    \item Сформировать множество вершин, у которых нет предшественников. Множество $D = \left( d_1, d_2 \dots d_i \right)$ где $d_i$ – номер работы, доступной для добавления в расписание (т.е. у которой нет предшественников в исходном графе).
    \item \label{METIS} В случае, если дополнительная дополнительное ограничение $CR$ - запустить алгоритм разбиения графа METIS.
    \item \label{itm:calcD} По жадному критерию $GC1$ выбирается работа из множества $D$ для размещения в расписании. Пусть выбранная работа – $d_i$.
          \begin{figure}[H]
              \ctikzfig{max_children}
              \caption{Выбор вершины в соответствии с жадным критерием $GC1$}
          \end{figure}
    \item Производится пробное размещение работы $d_i$. В случае задачи с дополнительным ограничение $CR$, задача ставится на процессор, которое определяет разбиение из пункта \ref{METIS}. В случае постановки задачи без дополнительных ограничений, задача ставится в соответствии с жадным критерием $GC2$.
    \item $d_i$ удаляется из списка размещенных работ и в графе $G$ удаляется соответствующая вершина и все дуги, исходящие из нее.
    \item Обновляется множество $D$. Если $D$ не пустое, то алгоритм переходит на пункт \ref{itm:calcD}.
\end{enumerate}
\subsubsection{Жадный критерий выбора работы $GC1$}
\begin{itemize}
    \item Максимальное количество потомков у работы
\end{itemize}
Такой выбор работы позволяет открыть максимально возможное количество кандидатов на следующую постановку задачи в расписание, а значит с минимальной вероятностью закрывает путь к оптимальному решению.
\subsubsection{Жадный критерий выбора места работы в расписании $GC2$} \label{sec:get_crit}
\begin{itemize}
    \item Среди доступных процессоров выбирается тот, при постановки на который время завершения работы на процессоре будет минимальным.
\end{itemize}
\subsubsection{Расчет времени начала работы} \label{sec:gap_filling}
При постановке задачи $d_i$ на процессор $p_i$:
\begin{enumerate}
    \item Рассматривается множество всех ее предшественников $Pr = \left( pr_1, pr_2, \ldots, pr_i \right)$
    \item Вычисляются времена завершения $pr_i$ на распределенный процессор.
    \item Пусть $pr_i$ распределена на процессор $p_k$. Если $p_k$ не равна $p_i$, к времени завершения работы $pr_i$ добавляется время передачи между $p_k$ и $p_i$. 
    \item После чего из получившегося набора берется максимум - "зависимость по задачам"
    \item После чего рассматриваются все пробелы (участки временной диаграммы между завршением $i$-й работы и началом $i+1$-й) на $p_i$, находящиеся после "зависимости по задачам", и, если длительность какого-либо пробела больше или равна времени выполнения $d_i$ на $p_i$ - задача помещается в пробел.
    \item Если данная процедура не нашла пробел - задача помещается в конец.
\end{enumerate}

\subsection{Жадный алгоритм, опирающийся на фиктивные директивные сроки}
\begin{enumerate}
    \item \label{METIS} В случае дополнительного ограничения $CR$ - строится взвешенное разбиение при помощи METIS.
    \item \label{first_addition} Листовые вершины добавляются в множество $D$.
    \item Для каждой вершины из $DL$ вычисляется ее фиктивный директивный срок. Первые листовые вершины, добавленные в пункте \ref{first_addition} получают директивные сроки $0$. Остальные вершины получают директивные сроки, равные разности минимального директивного срока среди потомков задачи и времени выполнения задачи. В случае дополнительного ограничения $CR$ также вычитается время передачи с процессора, на который распределен потом задачи и на который распределена задача.
    \item Задачи ставятся в расписание в порядке возрастания директивных сроков. В случае дополнительного ограничения $CR$ процессор выбирается из распределения, полученного в пункте \ref{METIS}. Для постановки задачи $NO$ Процессор выбирается с учетом дополнительного ограничения $GC2$ (пункт \ref{sec:get_crit}). Постановка происходит алгоритмом, описанным в \ref{sec:gap_filling}
\end{enumerate}
