\documentclass[12pt]{article}

% \usepackage{identfirst}

\usepackage[utf8]{inputenc}
\usepackage[T2A]{fontenc}
\usepackage[russian]{babel}

\usepackage{csquotes}
\usepackage{amsmath}

\usepackage{makecell}
\usepackage{tabularx}
\usepackage{float}
\usepackage[toc]{appendix}

% \usepackage{listings}
\usepackage[newfloat]{minted}

\usepackage[
backend=biber,
style=gost-numeric,
sorting=ynt
]{biblatex}


\usepackage[a4paper,top=20mm,bottom=20mm,left=30mm,right=10mm]{geometry}

\usepackage{tocloft}
\usepackage{caption}
\usepackage{subcaption}

\usepackage{mathtools}

% \usepackage{tikz}
\usepackage{pgf-umlcd}
\usepackage{tikzit}

\usepackage{pifont}
\usepackage{flowchart}

\renewcommand{\umlfillcolor}{white}
\renewcommand{\umldrawcolor}{black}
% \usepackage{underscore}

\renewcommand{\cftsecleader}{\cftdotfill{\cftdotsep}}

% \newenvironment{code}{\captionsetup{type=listing}}{}
\SetupFloatingEnvironment{listing}{name=Листинг}

\newcommand{\intro}[1]{
    \stepcounter{section}
    \section*{\hfillПРИЛОЖЕНИЕ \arabic{section}}
    \begin{center}
        \bf{#1}
    \end{center}
    \markboth{\MakeUppercase{#1}}{}
    \addcontentsline{toc}{section}{Приложение \arabic{section}. #1}
}

\usepackage{titlesec}

\setcounter{secnumdepth}{4}

\titleformat{\paragraph}
{\normalfont\normalsize\bfseries}{\theparagraph}{1em}{}
\titlespacing*{\paragraph}
{0pt}{3.25ex plus 1ex minus .2ex}{1.5ex plus .2ex}

\addbibresource{references.bib}

\input{block-schemas.tikzstyles}

\author{Савицкий Илья Павлович}
\title{Жадные алгоритмы для построения многопроцессорного списочного расписания}
\date{Москва, 2022}

% \pgfsetlayers{connectionlayers}

\begin{document}
\makeatletter
\begin{titlepage}
    \begin{center}
        \includegraphics[width=9cm]{imgs/msulogo.png}\\
        \small
        \centerline{Московский государственный университет имени М.В. Ломоносова}
        \centerline{Факультет вычислительной математики и кибернетики}
        \centerline{Кафедра автоматизации систем вычислительных комплексов}
        \centerline{}
        \Large
        \vfill
        {\@author}\\
        \null
        {\LARGE \bf
        \@title
        }\\
        \null \null
        {\large Курсовая работа}\\
        \null \null
    \end{center}
    \begin{flushright}
        {\bf Научный руководитель:}\\
        Доцент, к.т.н \\Костенко Валерий Алексеевич\\
        ~\\
        % {\bf Научный консультант:}\\
        % к.ф.-м.н.\\Иванов Иван Иванович\\
        \vfill
    \end{flushright}
    \centerline{Москва, 2022}
\end{titlepage}
\setcounter{page}{2}

\newpage
\begin{abstract}
    Построение многопроцессорного расписания это NP-трудная задача. Точный полиномиальный алгоритм для решения этой задачи неизвестен. В данной работе приводятся два возможных варианта решения задачи с дополнительным ограничением на количество передач при помощи жадных алгоритмов.
\end{abstract}
\newpage
\tableofcontents
% ----------------------------------------------------------------------------------------------------------------------
\newpage
\section{Введение}
Классическая задача построения расписания хорошо изучена и досканально описана в \cite{Coffman}. Поскольку данная задача принадлежит к классу NP-hard, не существует алгоритма, который за полиномиальное время даст точный ответ, но существуют алгоритмы, которые дают аппроксимированные результаты. Большинство таких алгоритмов резделябтся на две категории: \textit{конструктивные} и \textit{итерационные}. Из основных примеров можно выделить (большинство из них упомянуты в \cite{Kostenko_2017}):
\begin{itemize}
    \item Конструктивные алгоритмы
    \begin{enumerate}
        \item Алгоритмы, основанные на поиске максимального потока в сети
        \item Алгоритмы, основанные на методах динамического программирования
        \item Алгоитмы, основанные на методе ветвей и границ
        \item Жадные алгоритмы
        \item Жадные алгоритмы с процедурой ограниченного перебора
    \end{enumerate}
    \item Итерационные алгоритмы
    \begin{enumerate}
        \item Генетические алгоритмы
        \item Дифференциальная эволюция
        \item Алгоритм имитации отжига
        \item Алгоритм муравьиных колоний
    \end{enumerate}
\end{itemize}
\par
Конструктивные алгоритмы работают, строя и дополняя частичные расписания до тех пор, пока все работы не будут размещены. Итерационные же алгоритмы строят приближения расписания и оптимизируют их.
\par
В данной работе рассматриваются жадные алгоритмы с процедурой ограниченного перебора. Особенностью таких алгоритмов является баланс между двумя процессами построения расписания. Жадные стратегии строят расписание быстро, однако очень быстро могут зайти в тупик при построении расписания. В таком случае, если расписание строится си сильным отклонением от оптимального, процедура ограниченного перебора корректирует его.
% ----------------------------------------------------------------------------------------------------------------------
\newpage
\section{Цели и задачи этой работы}
Целью этой курсовой работы является разработка алгоритмов построения многопроцессорного расписания с дополнительными ограничениями на основе комбинации жадных алгоритмов. 

Для достижения указанной цели требуется:
\begin{enumerate}
    \item Провести обзор алгоритмов построения списочных расписаний с целью выявления жадных критериев и схем ограниченного перебора которые могут быть модифицированы для решения данной задачи.
    \item Разработать алгоритмы.
    \item Реализовать алгоритмы.
    \item Провести исследование свойств алгоритма. 
\end{enumerate}
% ----------------------------------------------------------------------------------------------------------------------
\newpage
\section{Задача построения списочных расписаний}
\subsection{Дано}
\begin{enumerate}
    \item Ориентированный граф работ $G$ без циклов, в котором дуги - зависимости по данным, а вершины - задания. Вершин $n$, дуг $m$
    \item Вычислительная система, состоящая из $p$ различных процессоров
    \item Матрица $C_{ij}$ длительности выполнения работ на процессорах, $i=1 \dots n, j=1 \dots p$. Каждая строка этой матрицы - длины выполнения $n$-й задачи на $p$ процессорах. 
    \item Матрица $D_{kl}$ передач данных между процессорами, $k=1 \dots p, l = 1 \dots p, D_{kk} = 0$. $D_{ij}$-й элемент этой матрицы - время пеердачи данных между процессорами $i$ и $j$.
\end{enumerate}
\subsection{Определение расписания}
Расписание программы определено, если заданы:
\begin{enumerate}
    \item Множества процессоров и работ
    \item Привязка - всюду определенная на множестве работ функция, которая задает распределение работ по процессорам
    \item Порядок - заданные ограничения на последовательность выполнения работ и является отношением частичного порядка, удовлетворяющим условиям ацикличности и транзитивности. Отношение порядка на множестве работ, распределенных на один процессор, является отношением полного порядка.
\end{enumerate}
\subsubsection{Способы представления расписаний}
Пусть дан следующий граф потока данных:
\begin{figure}[H]
    \ctikzfig{schedule-graph}
    \caption{Граф $G$ потока данных}
\end{figure}
Пусть в оптимальном расписании работы $1$, $10$, $4$, $6$ будут поставлены на $Pr1$, $2$, $5$, $8$ - на $Pr2$, а $3$, $9$ - на $Pr3$. Рассмотрим как такое расписание будет выглядеть в различных формах:
\begin{enumerate}
    \item Графическая форма представления
    \begin{figure}[H]
        \ctikzfig{schedule-graphical-form}
        \caption{Графическое представление расписания}
    \end{figure}
    В такой форме представления расписания каждой задаче сопоставляется пара из номера процессора и порядкового номера работы на процессоре, а так же секущие дуги, если задачи зависят друг от друга. 
    \item Временная диаграмма
    \begin{figure}[H]
        \ctikzfig{schedule-time-diagram}
        \caption{Представление расписания в виде временной диаграммы}
    \end{figure}
    В такой форме представления расписания каждой задаче сопоставляется пара из номера процессора и временис старта задачи на процессоре.
\end{enumerate}
Доказано, что эти формы полностью эквивалентны и, имея одну, возможно построить другую. В предложенном решении расписание строится в виде временной диаграммы.

\subsection{Требуется}
\begin{enumerate}
    \item Построить расписание $HP$, то есть для $i$-й работы определить время начала ее выполнения $s_i$ и процессор $p_i$ на которм она будет выполняться
    \item В расписании требуется минимизировать время выполнения набора работ, данных в графе $G$
    \item В задаче так же присутствуют дополнительные ограничения, котрым расписание обязано удовлетворять.
\end{enumerate}
Ограничения на корректность расписания следующие:
\begin{enumerate}
    \item Каждый процессор может одновременно выполнять не больше одной работы;
    \item Прерывание работ недопустимо, перенос частично выполненной работы на другой процессор недопустим;
    \item Если между двумя работами есть зависимость на данным, то между завершением работы-отправителя и сиартом работы-получателя должен быть интервал времени, не меньший чем задержка на передачу данных между ними (с учетом привязки работ к процессорам и маршрута передачи данных).
\end{enumerate}
\subsection{Различные постановки задачи} \label{sec:crit}
\begin{enumerate}
    \item Задача с однородными процессорами (длительность выполнения работы не зависит от того, на каком процессоре она выполняется) и дополнительными ограничениями на количество передач:
          \begin{itemize}
              \item $CR = \frac{m_{ip}}{m} < 0.4$, где $m_{ip}$ - количество передач данных между работами на каждый процессор. (Cut ratio)
              \item $CR2 = \frac{m_{2edg}}{m} < 0.05$, где $m_{2edg}$ - количество дуг, начальный и конечный узлы которых назначены на процессоры, не соединенных напрямую (Cut ratio 2)
          \end{itemize}
    \item Задача с однородными процессорами и дополнительным ограничением сбалансированности распределения работ:
          \begin{itemize}
              \item $BF = \lceil 100 \cdot \left( \frac{a_{max} \cdot p}{n} - 1 \right) \rceil < 10$, где $a_{max}$ - наибольшее, по всем процессорам, количество работ на процессоре (Balance factor)
          \end{itemize}
    \item Задача с неоднородными процессорами, но без дополнительных ограничений на расписание
\end{enumerate}
% ----------------------------------------------------------------------------------------------------------------------
\newpage
\section{Обзор предметной области}
% Обзоры других алгоритмов приведены в \cite{Shakhbazyan_1981,Davis_2011}
\subsection{Критерии обзора}
Ниже приведены критерии, по которым будут рассматриваться и сравниваться алгоритмы
\begin{enumerate}
    % \item Насколько сильно рассматриваемая в статье задача отличается от решаемой. Можно ли взять алгоритм, описанный в статье за базу для алгоритма для решения данной задачи.
    % \item Точность решений, строимых рассмотренными алгоритмов
    % \item Порядок сложности алгоритма, насколько он масштабируемый.
    % \item На данных какой размерности протестирован алгоритм. Время его работы.
    \item Возможность модифицировать алгоритм под поставленную задачу. Из необходимых модификаций требуется добавить возможность учета дополнительного критерия межпроцессорных передач и возможность учета затрат на межпроцессорные передачи.
    \item Возможность масштабирования алгоритма. 
\end{enumerate}
\subsection{Конструктивные алгоритмы}
\subsubsection{Жадные алгоритмы}
Жадные алгоритмы подразумевают декомпозицию задачи на ряд более простых подзадач. На каждом шаге решение принимается исходя из принципа получения оптимального решения для очередной подзадачи. То есть, на каждом шаге алгоритм делает выбор, оптимальный с точки зрения получения решения очередной подзадачи, предполагая, что эти локально-оптимальные решения приведут к приемлемому решению задачи. Какие-либо жадные стратегии, гарантированно получающие оптимальное расписание, на настоящий момент времени неизвестны, за исключением небольшого числа вариантов задач составления расписаний не принадлежащих к классу NP-полных. Например, известен жадный алгоритм, получающий точное решение для задачи обслуживания одним процессором максимального числа работ из заданного набора работ с фиксированными сроками начала и окончания \cite{Cormen}. Набор локальных критериев оптимизации сильно зависит от класса архитектуры. Для архитектур, в которых возможно последействие (распределяемый в расписание рабочий интервал оказывает влияние на времена инициализации ранее распределенных рабочих интервалов) возникает проблема выбора локальных критериев оптимизации, позволяющих учесть эффект последействия (на настоящий момент времени какие-либо обоснованные решения этой проблемы не известны). Кроме того, единого локального критерия (или набора и способа их использования), приводящего к наилучшему конечному результату, для решения всех подзадач не существует. Более того, при усложнении архитектуры набор и способ использования локальных критериев оказывает более сильное влияние на конечный результат. Таким образом, применение жадных алгоритмов для составления расписаний классом архитектур без последействия или даже без разделяемых ресурсов, если их влияние на значение функции построения временной диаграммы не может быть локализовано, а также проблемой выбора критериев оптимизации индивидуально для каждой подзадачи.

При построении расписания жадным алгоритмом для каждой заадчи необходимо определить два параметра:
\begin{enumerate}
    \item Привязку задачи к процессору $p_i$
    \item Время старта задачи на процессоре $s_i$
\end{enumerate}
Каждый из этих параметров может быть определен своим жадным критерием. Время старта (как показано в \cite{Kalashnikov_2004}) единственным образом определяется из порядка работ на процессоре. Следовательно, имеет роль порядок, в котором работы добавляются в расписание.

Таким образом, для построения расписания достаточно определить:
\begin{enumerate}
    \item Привязку задачи к процессору $p_i$
    \item Ее номер в очереди на добавление в расписание $q_i$
\end{enumerate}

\subsubsection{Алгоритмы, основанные на методе динамического программирования}

Алгоритмы динамического программирования разбивают сложную задачу на более простые подзадачи и находят обратную связь между их оптимальными решениями. Алгоритмы из этой области могут предоставлять глобальные оптимальные решения, но их недостатком является неполиномиальная сложность. В частности, с точки зрения NP-сложных задач планирования сложность этих алгоритмов является экспоненциальной функцией размера входных данных. Кроме того, для нахождения обратной зависимости между оптимальными решениями и подзадачами необходима модель аналитической системы. Это означает, что алгоритм не подходит для решения данной задачи, так как имеет высокую вычислительную сложность \cite{Dynamic_prog}.

\subsubsection{Алгоритмы, основанные на методе ветвей и границ}

Алгоритм ветвей и границ является эффективным методом решения производных задач оптимизации. Этот метод делит пространство потенциальных решений на различные области, дает точные оценки их значения по отношению к целевой функции и сокращает ненужные участки, где невозможно найти оптимальное решение. Хотя глобальные оптимальные решения могут быть получены с помощью этого метода, его сложность для большинства задач комбинаторной оптимизации неполиномиальна. Особенно для NP-сложных задач планирования сложность является факториальной функцией размера входных данных. Следовательно, эти алгоритмы не подходят для решения проблемы.

Алгоритм, обсуждаемый в \cite{Rahman2009BranchAB}, был протестирован с использованием наборов данных с числом процессоров до 16 и графов, содержащих до 100 вершин. Результаты показали ожидаемый экспоненциальный рост времени работы.

\subsubsection{Алгоритмы, основанные на нахождении максимального потока в сети}

Алгоритмические методы составления многопроцессорных расписаний включают поиск максимального потока в транспортной сети, которые, по сути, переводят процесс построения расписания в поиск максимально возможного потока в указанной сети. Указанная сеть строится на основе набора задач, процессоров и параметров исходного задания. После построения сети выполняется процесс поиска максимального потока. Затем, с помощью значений потока в сети, возможно построить многопроцессорное расписание \cite{MAGIROU1989351}. Этот конкретный алгоритм подходит только для задач, допускающих прерывания в вычислительной системе, поэтому в данной задаче его нельзя использовать.

% \subsection{Итерационные алгоритмы}

% \subsubsection{Генетические алгоритмы}

% Такие алгоритмы используют селекцию, кроссинговер и мутацию, чтобы оптимизировать набор решений, называющийся популяцией, при этом сохраняя среди них достаточное разнообразие чтобы не находить решения в локальных минимумах.

% Недостатком этого типа алгоритмов является отсутствие масштабируемости. По мере увеличения использования становится все труднее управлять системой или масштабировать ее для различных задач. Результаты исследования в статье демонстрируют, что алгоритм не имеет высокой производительности при использовании больших объемов данных. Разобранный в \cite{Sheikh2016AnET} алгоритм смог обработать задачу, содержащую 1000 заданий, за 1,5 часа на процессоре с частотой 2 ГГц. Хотя это может показаться длительным временем выполнения, это в значительной степени связано с тем, что генетические и эволюционные алгоритмы требуют наличия множества решений, доступных в популяциях для реализации.

% \subsubsection{Алгоритм имитации отжига}

% Алгоритмы имитации отжига хорошо подходят для нахождения приближенных решений NP-сложных задач комбинаторной оптимизации. Эти алгоритмы позволяют решать проблемы с помощью стохастических шагов, гарантируя получение наилучшего решения \cite{Kalashnikov_2004}.

% Имитация отжига может эффективно обрабатывать большие наборы данных, так как этот алгоритм работает с одним решением. В отличие от генетических и эволюционных алгоритмов, это делает его более масштабируемым.

% \subsubsection{Алгоритм муравьиных колоний}

% Муравьиные алгоритмы - это методы оптимизации, которые используют положительную и отрицательную обратную связь для поиска оптимального пути \cite{Shtovba_2005}. Они находят широкое применение в различных задачах оптимизации, но могут столкнуться с проблемой преждевременной сходимости, что может привести к неоптимальным результатам. Важным аспектом применения муравьиных алгоритмов является тщательная настройка и учет особенностей конкретной задачи, чтобы достичь наилучшего результата.

\subsection{Выводы из обзора предметной области}
% Итерационные алгоритмы работают путем создания аппроксимированного варианта решения и последующего его улучшения. Однако, большинство из них рандомизированные и, следовательно, из нескольких различных запусков теоретически возможно получить различные расписания (несмотря на то то они все сходятся). Более того, многие из таких алгоритмов плохо масштабируемы. Муравьиные алгоритмы разобраны в \cite{Shtovba_2005}, имитации отжига - в \cite{Kirkpatrick_1983}.
% \begin{table}[htbp!]
%     \begin{tabularx}{\textwidth}{ | X | l | X | X | }
%         \hline
%         Название алгоритма            & Рандомизированность & Класс алгоритмa  & Возможность  масштабирования \\
%         \hline
%         Генетические алгоритмы        & Рандомный           & Итерационный    & +/-                          \\
%         \hline
%         Алгоритм имитации отжига      & Рандомный           & Итерационный    & +                            \\
%         \hline
%         Муравьиные алгоритмы          & Рандомный           & Итерационный    & -                            \\
%         \hline
%         Жадные стратегии              & Детерминированный   & Конструктивный  & +                            \\
%         \hline
%         Динамическое программирование & Детерминированный   & Конструктивный  & -                            \\
%         \hline
%         Ветви и границы               & Детерминированный   & Конструктивный  & -                            \\
%         \hline
%         Максимальный поток            & Детерминированный   & Конструктивный & -                            \\
%         \hline
%     \end{tabularx}
%     \caption{Существующие алгоритмы}
%     \label{tbl:review}
% \end{table}

\newcolumntype{Y}{>{\centering\arraybackslash}X}

\begin{table}[!htbp]
    \begin{tabularx}{\textwidth}{  c | Y | Y }
        Название алгоритма                   & Возможность модификации алгоритма & Возможность масштабирования алгоритма \\
        \hline
        Метод ветвей и границ                & \ding{51}                         & \ding{55}                             \\
        Метод динамического программирования & \ding{51}                         & \ding{55}                             \\
        Алгоритм поиска максимального потока & \ding{55}                         & \ding{51}                             \\
        Жадные алгоритмы                     & \ding{51}                         & \ding{51}                             \\
    \end{tabularx}
    \caption{Существующие детерминированные алгоритмы}
    \label{tbl:review}
\end{table}

В таблице \ref{tbl:review} представлены сокращенные результаты обзора.

В результате обзора предметной области, под критерии масштабируемости и соответствия задаче подходит жадный алгоритм. По результатам не было найдено работ, точно соответствующим постановке задачи, поэтому предложенный подход требуется модифицировать.
% Обзоры этих алгоритмов для схожих задач представлены в \cite{Coffman,Davis_2011,Shakhbazyan_1981}


% ----------------------------------------------------------------------------------------------------------------------
\newpage
\section{Алгоритм построения расписания}
\subsection{Алгоритмы построения расписания}
\subsubsection{Описание жадных алгоритмов построения многопроцессорного расписания} \label{algo_template}
Жадные алгортмы, представленные в данной работе, построены по следующей схеме:
\begin{enumerate}
    \item \label{GC1} Выбрать работу для постановки в расписание.
    \item \label{GC2} Выбрать процессор и время начала выполнения задачи из п.\ref{GC1} на выбранном процессоре.
    \item Поставить работу на процессор.
    \item Остановиться, если все задачи поставлены, иначе перейти к п.\ref{GC1}
\end{enumerate}

\subsubsection{Жадный алгоритм} \label{Greedy_GC1}
Жадный алгоритм следует общей схеме, описанной в \ref{algo_template}

Эту схему можно уточнить путем выбора критерия отбора в п.\ref{GC1} и критерия выбора процессора и начала времени выполнения в п.\ref{GC2}. Для постановки задачи с дополнительными ограничениями, такого как $CR$, так же может быть неудача в постановки задачи в раписание, в случае невозможности постановки без нарушения дополнительного ограничения. В таком случае, алгоритм завершается.

Задача $d_i$ наывается \textbf{доступной для постановки} в расписание, в случае, если либо у нее нет предшественников, либо все ее предшественники уже постановлены в расписание. Назовем множеством всех доступных для постановки задач $D = \left( d_0, d_1, \ldots, d_n \right)$.

Жадный алгоритм выбирает задачу для постановки по следующему критерию: пусть $Succ(d)$ - функция, определяющая количество непосредственных последователей работы в графе. Тогда $\forall d_j \in D, d_j \neq d_i: Succ(d_j) < Succ(d_i)$.

\begin{figure}[!htbp]
    \ctikzfig{max_children}
    \caption{}
    \label{fig:max-children}
\end{figure}
На рисунке \ref{fig:max-children} $D = \left( x_1, x_2 \right)$, из которых $Succ(x_1) = 2, Succ(x_2) = 1$. На постановку будет выбрана вершина $x_1$. 

Для постановки с дополнительным ограничением $CR$ жадный алгоритм берет распределение работ по процессорам из специального алгоритма распределения (описание алгоритма приведено в разделе \ref{METIS}), поэтому выбор процессора в п.\ref{GC2} всегда заранее детерминирован. Выбор начала выполнения работы в расписание производится в соответствии с алгоритмом постановки задачи на процессор (описание алгоритма приведено в разделе \ref{gap_filling}).

Для постановки без дополнительных ограничений, жадный алгоритм производит пробную постановку на каждый процессор и выбирает процессор с самым ранним временем завершения работы с учетом алгоритма постановки задачи на процессор (описание алгоритма приведено в разделе \ref{gap_filling}). 

\subsubsection{Жадный алгоритм с EDF эвристикой} \label{Greedy_EDF}
Данный алгоритм следует общей схеме, описанной в пункте \ref{algo_template}, однако отличается от алгоритма, описанного в пункте \ref{Greedy_GC1} критерием выбора работы на постановку.

Эвристика ``саммый ранний директивный срок первый'' (earliest deadline first, или \textbf{EDF}) упорядочивает работы по возрастанию директивных сроков и выбирает работу с нименьшим директивным сроком на постановку. Однако, постановка задачи не предполагает у задач директивных сроков, поэтому в данном алгоритме у каждой работы строятся фиктивные директивные сроки.

В случае, если существует директивный срок всего расписания $d$, то директивный срок $d_A$ работы $A$ может быть рассчитан следующим образом (при известном распределении работ на процессоры):
\begin{enumerate}
    \item \label{find_path} Найти длиннейший путь в графе потока управления от $A$ до работы $S: Succ(S) = \emptyset$.
    \item Рассчитать длину этого пути. Длина пути равна сумме всех передач задержек данных и времен выполнения работ. Задержки передач данных известны, так как известно распределение работ на процессоры. Пусть длина этого пути равна $p$.
    \item $d_A \coloneqq d - p$
\end{enumerate}
Видно, что работа $A$ должна завершиться до $d_A$; иначе путь, найденный в п.\ref{find_path} завершится позже, чем $d$, даже если процессоры ни имеют никакой другой нагрузки, кроме этих работ.

Даже без изветсного директивного срока расписания, EDF эвристика все еще может быть использована для сортировки работ по уменьшению ``потенциальной длины пути  до конца расписания'', учитывая, что расписание всегда завершится какой-либо работой $S : Succ(S) = \emptyset$. Также, нет необходимости вводить настоящие директивные сроки, в которые работы должны быть завершены. Таким образом, можно выставить директивный срок распсиания в $0$, и получить формальные директивные сроки по алгоритму, представленному выше. Такие директивные сроки могут быть отрицательными, что не препятствуует сортировать работы по их возрвствнию.

Описанный алгоритм без модификаций применим к задаче с дополнительным ограничением $CR$, поскольку распределение работ на процессоры может быть рассчитано заранее, и поэтому, время задержек межпроцессорных передач известно заранее. Для данных с однородными процессорами строится взвешенное разбиение (\ref{METIS}), для постановки с неоднородными процессорами строится невзвешенное разбиение, которое впоследствии улучшается алгоритмом локальной оптимизации (\ref{partition_optimization}). 

Однако, для постановки задачи без дополнительных ограничений привязка задач к процессорам заранее неизвестна, поэтому, для вычисления директивных сроков не учитываются задержки межпроцессорной передачи данных, а время выполнения данной задачи считается усредненным по всем процессорам. Например, если в системе три процессора, на которых задача выполняется 1, 1 и 4 у.е., то для рассчета директивного срока время выполнения данной задачи считается $(1+1+4)/3 = 2$. Такая аппроксимация не нарушает работу алгоритма, поскольку директивные сроки требуется только для сортировки работ.

Жадный алгоритм с EDF эвристикой начинается с вычисления фиктивных директивных сроков, после чего выполняется цикл, описанный в \ref{algo_template}.

Еще не добавленная работа с минимальным фиктивным директивным сроком выбирается как очередной кандидат на добавление в расписание. 

Аналогично алгоритму, описанному в \ref{Greedy_GC1}, для задачи с дополнительным ограничением $CR$ для выбора процессора для постановки очередной задачи используется распределение, построенное алгоритмом распределения задач на процессоры (\ref{METIS}), а для постановки без дополнительных ограничений производит пробную постановку на каждый процессор с самым ранним временем завершения работы с учетом алгоритма постановки задачи на процессор (\ref{gap_filling}).

\subsection{Вспомогательные алгоритмы}

\subsubsection{Алгоритм локальной оптимизации разбиения} \label{partition_optimization}

Этот алгоритм используется только для постановки $CR$ с неоднородными процессорами. Подразумевается, что процессоры упорядочены по возрастанию производительности, то есть для любых процессоров $P_1, P_2$, если работа $A$ выполняется на процессоре $P_1$ дольше, чем на $P_2$, то и работа $B$ выполняется на процессоре $P_1$ дольше, чем на $P_2$. Таким образом, если переназначить работу с $P_1$ на $P_2$, то время выполнения расписания сократится.

С примерно равным количеством работ на всех процессорах на невзвешенном разбиении METIS, самые медленные процессору будут самыми загруженными. Задача данного алгоритма - "разгрузить" самые загруженные процессоры путем перемещания раот с него на менее загруженные процессоры. 

Алгоритм имет следующую структуру:
\begin{enumerate}
    \item Выбрать самый загруженный процессор $P_1$
    \item Для каждой работы $A$, поставленной на $P_1$, в порядке убывания времени выполнения:
    \begin{enumerate}
        \item \label{item:choose_proc} Выбрать самый быстрый процессор $P_2$ из процессоров, удовлетворяющих следующее условие: если перенести работу $A$ с $P_1$ на $P_2$, то $\max(\text{загрузка} P_1, \text{загрузка} P_2)$ уменьшается и выполняется ограничение $CR$
        \item Если такой $P_2$ был найден, то перенести эту задачу и перейти к пункту \ref{item:choose_proc}; иначе рассмотреть следующую по времени выполнения задачу на $P_1$.
    \end{enumerate}
    \item Если задачи на $P_1$ кончились, то остановить алгоритм.
\end{enumerate}

\subsubsection{Алгоритм распределения задач на процессоры} \label{METIS}
В качестве алгоритма распределения задач на процессоры был выбран алгоритм разбиения графа на подграфы METIS \cite{Karypis2011}. 

Для построения распределения работ на процессоры запускается алгоритм кластеризации графа с количеством кластеров, равным количеству процессоров, после чего каждый кластер распределенных задач присваивается одному процессору. 

Для задачи с дополнительным ограничением $CR$ используется взвешенное распределение METIS, где каждой вершине придается вес, равный времени выполнения задачи на процессоре. Поскольку в этой постановке проуессоры равный, конкретный процессор в которого берется время выполнения не имеет значения.

Разбиение, лучшее по балансу кластеров, может нарушать ограничение $CR$, в случае, если большие группы плотно взаимодействующих работ распределятся на разные процессоры. Эта проблема решается варьированием параметра \texttt{ufactor} алгоритма METIS, который контролирует отношение максимального количества работ в подграфе к вреднему количеству работ в подграфе. Другими словами, \texttt{ufactor} позволяет контролировать дизбаланс в количестве вершин в кластере. С увеличением этого параметра, $CR$ понижается. Для генерации распределения, удовлетворяющего дополнительное ограничение $CR$, достаточно генерировать распределения с постепенным увеличением \texttt{ufactor} до тех пор, пока очередное распределение не выполнит $CR$.

\subsubsection{Алгоритм постановки задачи на процессор} \label{gap_filling}
При постановке задачи на заданный процессор достаточно вычислить время начала выполнения задачи $t$ такое, чтобы кждое частичное расписание после добавления оставалось корректным. Начальное время $t$ выбирается как минимальное время, удовлетворяющее следующим условиям:
\begin{enumerate}
    \item Все передачи данных от предшествующих задач завершились до $t$.
    \item Существует свободный интревал времени (простой процессора или после завершения последней поставленной задачи), начинающийся в $t$ и длительностью, больший или равный времени выполнения работы, в который не выолняется ни одназ работа. В некоторых случаях, задача будет поставлена до начала другой, не свзяанной с ней задачей, в случае, если времени простоя достаточно. В сложных графах потока управления, такие простои возникают в частичных расписаниях часто, а значит всегда есть смысл их заполнять.
\end{enumerate}



% ----------------------------------------------------------------------------------------------------------------------
\newpage
\section{Программная реализация алгоритма}
\subsection{Описание кода программной реализации}
Код реализации выложен на \mintinline{bash}{C++} в репозитории \cite{Repository}. Диаграмма калссов реализации представлена на рисунке \ref{fig:UML}.

\begin{figure}[!htbp]
    \centering
    \includegraphics[width=0.8\textwidth]{imgs/final_UML.drawio.png}
    \caption{UML-диаграмма реализации}
    \label{fig:UML}
\end{figure}

Среди представленных классов:
\begin{itemize}
    \item \mintinline{C++}{ScheduleData} - класс, хранящий в себе входные данные и выполняющий всю необходимую их предобработку.
    \item \mintinline{C++}{TimeDiagram} - класс, хранящий в себе частичное или полное расписание.
    \item \mintinline{C++}{PlacedTask} - класс, хранящий в себе информацию о поставленной в расписание работе.
\end{itemize}

Жадные алгоритмы реализованы в функциях, не инкапсулированных в классах:
\begin{itemize}
    \item \mintinline{C++}{construct_time_schedule()} - жадный алгоритм с жадным критерием.
    \item \mintinline{C++}{greedy_EDF_heuristic()} - Жадный алгоритм с EDF эвристикой.
\end{itemize}

В репозиторий включены следубщие библиотеки:
\begin{enumerate}
    \item \mintinline{bash}{METIS} 5.1.0 \cite{METIS_lib} - библиотека для разбиения графов.
    \item \mintinline{bash}{json} 3.11.2 \cite{json_lib} - библиотека для работы с форматом JSON. Используется для составления выходных файлов.
    \item \mintinline{bash}{toml11} 3.7.1 \cite{toml11_lib} - библиотека для работы с форматом TOML. Используется для чтения конфигурационных файлов.
\end{enumerate}

Также, у реализации есть зависимость, не включенная в репозиторий - \mintinline{bash}{boost} 1.80 \cite{boost_framework}. Для сборки проекта используется \mintinline{bash}{CMake}. Инструкция по сборке приведена в листинге \ref{lst:build}. Для сборки документации (на английском) используется \mintinline{bash}{Doxygen}. Инфструкция по сборке документации приведена в листинге \ref{lst:docs}

\begin{listing}[!htbp]
    \begin{minted}{bash}
        mkdir build
        cd build
        cmake ..
        make
    \end{minted}
    \caption{Сборка программной реализации}
    \label{lst:build}
\end{listing}

\begin{listing}[!htbp]
    \begin{minted}{bash}
        doxygen Doxyfile
    \end{minted}
    \caption{Сборка документации}
    \label{lst:docs}
\end{listing}


\subsection{Описание интерфейса программной реализации}
\subsubsection{Параметры командной строки}
Из исходного кода реализации алгоритма собираетяс утилита, с интерфейсом, описанным в листинге \ref{lst:template} и таблице \ref{tbl:command-line-parameters}. 
\begin{listing}[!htbp]
    \begin{minted}[breaklines]{bash}
        opts <algorithm_name> --input <input_file> --output <output_file> --conf <config_file> --log <log_level>
    \end{minted}
    \caption{Шаблон запуска утилиты построения расписания}
    \label{lst:template}
\end{listing}

\begin{table}[!htbp]
    \centering
    \begin{tabularx}{\textwidth}{|c|X|}
        \hline
        Имя                      & Описание                                     \\
        \hline
        \texttt{algorithm\_name} & Название алгоритма для построения расписания \\
        \hline
        \texttt{input}           & Путь к файлу с входными данными              \\
        \hline
        \texttt{output}          & Путь к файлу с выходными данными             \\
        \hline
        \texttt{conf}            & Путь к файлу с конфигурацией                 \\
        \hline
        \texttt{log}             & Уровень логирования                          \\
        \hline
    \end{tabularx}
    \caption{Параметры командной строки программы}
    \label{tbl:command-line-parameters}
\end{table}
\subsubsection{Описание конфигурационных файлов}
В качестве формата конфигурационных файлов был выбран формат \texttt{toml}. Пример конфигурационного файла приведен в листинге \ref{lst:config-file} и таблице \ref{tbl:config-file-parameters}. Конфигурационный файл содержит два раздела:
\begin{itemize}
    \item Раздел \texttt{[general]}, отвечающий за общие параметры построения расписания.
    \item Раздел \texttt{[greedy]}, отвечающий за параметры, относящиеся только к жадному алгоритму.
\end{itemize}

\begin{table}[!htbp]
    \centering
    \begin{tabularx}{\textwidth}{|c|X|}
        \hline
        Поле                & Описание                                                                                                                                      \\
        \hline
        \texttt{criteria}   & Критерий, дополнительное ограничение котрого будет выполняться (CR / NO)                                                                      \\
        \hline
        \texttt{CR\_bound}  & Верхняя граница ограничения $CR$ (если используется)                                                                                          \\
        \hline
        \texttt{inp\_class} & Класс типа входных данных
        \begin{itemize}
            \item \texttt{class\_1} для постановки с однородными процессорами
            \item \texttt{class\_general} для постановки с неоднородными процессорами
        \end{itemize}                                                                                            \\
        \hline
        \texttt{cr\_con}    & Переключение жадного критерия в жадном алгоритме с жадными критериями с максимального количества потомков на максимальное количество предков. \\
        \hline
    \end{tabularx}
    \caption{Параметры конфигурационного файла.}
    \label{tbl:config-file-parameters}
\end{table}
\begin{listing}
    \begin{minted}[linenos]{toml}
        [general] 
        criteria = "BF" 
        CR_bound = 0.4 
        inp_class = "class_1" 
        
        [greedy] 
        cr_con = false
    \end{minted}
    \caption{Пример конфигурационного файла}
    \label{lst:config-file}
\end{listing}

\subsubsection{Описание выходных файлов}
В качестве формата выходных файлов был выбран формат \texttt{json}. Пример конфигурационного файла приведен \ref{lst:output-file} и таблице \ref{tbl:output-file-fields}. Конфигурационный файл содержит информацию о характеристиках построенного расписания, а так же информацию о привзяках и порядке постановке работ на процессорах.
\begin{table}[!htbp]
    \centering
    \begin{tabularx}{\textwidth}{|c|X|}
        \hline
        Поле                & Описание                                                                                                                                                \\
        \hline
        \texttt{CR}         & Значение ограничения $CR$ построенного расписания.                                                                                                      \\
        \hline
        \texttt{algo\_time} & Время выполнения алгоритма, в миллисекундах                                                                                                             \\
        \hline
        \texttt{criteria}   & Дополнительное ограничение, используемое для построения рапсисания                                                                                      \\
        \hline
        \texttt{nodes}      & Количество работ во входном графе.                                                                                                                      \\
        \hline
        \texttt{time}       & Время выполнения построенного расписания                                                                                                                \\
        \hline
        \texttt{procs}      & Словарь с номерами процессоров в качестве ключей и массивами постанленных на соответствующий процессор работами. Каждая поставленная работа состоит из:
        \begin{itemize}
            \item \texttt{task\_dur} - время выполнения работы на поставленный процессор.
            \item \texttt{task\_no} - идентификатор работы.
            \item \texttt{task\_start} - время начала выполнения работы на процессоре.
        \end{itemize}                                                                                                  \\
        \hline
    \end{tabularx}
    \caption{Поля выходного файла}.
    \label{tbl:output-file-fields}
\end{table}

\begin{listing}[!htbp]
    \begin{minted}[linenos, mathescape=true, escapeinside=||]{json}
        {  
            "CR": 0.3221312,  
            "algo_time": 300, 
            "criteria": "CR", 
            "nodes": 2000, 
            "procs": { 
                "0": [ 
                    { 
                        "task_dur": 5, 
                        "task_no": 1202, 
                        "task_start": 0 
                    }, 
                    { 
                        "task_dur": 3, 
                        "task_no": 1608, 
                        "task_start": 5 
                    },
                    |\ldots|
                ], 
                "1": [ 
                    |\ldots|
                ], 
                |\ldots|
            }, 
            "time": 2211 
        }
    \end{minted}
    \caption{Пример выходного файла}
    \label{lst:output-file}
\end{listing}


% ----------------------------------------------------------------------------------------------------------------------
\newpage
\section{Экспериментальное исследование алгоритма}
\subsection{Подбор значений}

\subsection{Исследование свойств алгоритма}

% ----------------------------------------------------------------------------------------------------------------------
\newpage
\section{Заключение}
В ходе выполнения курсовой работы были достигнуты все ее цели, а именно:
\begin{enumerate}
    \item Проведен обзор существующих решений задач. Произведено сравнений других стратегий с жадными критериями и ограниченным перебором. На основе обзора произведен выбор подхода, основанного на комбинации жадных стратегий и ограниченного перебора. 
    \item Разработан и реализован алгоритм.
    \item Произведено исследование свойств алгоритма.
\end{enumerate}

При исследовании алгоритма были подобраны оптимальные параметры алгоритма и определены направления дальнейшего улучшения и исследования алгоритма.

% ----------------------------------------------------------------------------------------------------------------------
\newpage
% \printbibliography
\section{Список литературы}

\begin{enumerate}
    \item Held M., Karp R. M. A dynamic programming approach to sequencing problems //Journal of the Society for Industrial and Applied mathematics. – 1962. – V. 10. – №. 1. – P. 196-210.
    \item Coffman E. G. Computer and job-shop scheduling theory. — Nashville, TN : John Wiley
    \& Sons, 02.1976. — ISBN 0471163198.
    \item Шахбазян К. В., Тушкина Т. А. Обзор методов составления расписаний для многопроцессорных систем //Записки научных семинаров ПОМИ. – 1975. – Т. 54. – №. 0. – С. 229-258.
    \item Magirou V. F., Milis J. Z. An algorithm for the multiprocessor assignment problem //Operations research letters. – 1989. – V. 8. – №. 6. – P. 351-356.
    \item Cormen T. H. et al. Introduction to algorithms. – MIT press, 2022.
    \item Калашников А. В. Алгоритмы оптимизации расписаний, основанные на исправлении
    неоптимальных фрагментов. — 2004.
    \item Rahman M. Branch and Bound Algorithm for Multiprocessor Scheduling. – 2009 - URL: https://www.diva-portal.org/smash/record.jsf?pid=diva2\%3A518609\&dswid=-4391.
    \item Karypis G. METIS and ParMETIS // Encyclopedia of Parallel Computing / edited by D.
    Padua. — Boston, MA : Springer US, 2011. — P. 1117—1124. — ISBN 978-0-387-09766-4. —
    DOI: $10.1007/978-0-387-09766-4\_500$.
    \item Костенко В. А. Алгоритмы комбинаторной оптимизации, сочетающие жадные стратегии и ограниченный перебор //Известия Российской Академии наук. Теория и системы управления. – 2017. – №. 2. – С. 48-56.
    \item Canon L. C., Sayah M. E., Héam P. C. A comparison of random task graph generation methods for scheduling problems //Euro-Par 2019: Parallel Processing: 25th International Conference on Parallel and Distributed Computing, Göttingen, Germany, August 26–30, 2019, Proceedings 25. – Springer International Publishing, 2019. – P. 61-73.
    \item Boost C++ libraries. — URL: https://www.boost.org/ (дата обр. 02.04.2023).
    \item JSON parsing library. — URL: https://github.com/ nlohmann / json (дата обр.
    02.04.2023).
    \item METIS library. — URL: https://github.com/KarypisLab/METIS (дата обр. 10.04.2023).
    \item multiprocessor-scheduling. — URL: https://github.com/ipsavitsky/greedy-scheduling
    (дата обр. 23.04.2022).
    \item TOML parsing library. — URL: https://github.com/ToruNiina/toml11 (дата обр.
02.04.2023).

\end{enumerate}

\appendix

% \newpage
% \intro{Классы входных данных}
% В данных, присланных от Хуавей существует разделение на 2 класса.
\begin{enumerate}
    \item Первый класс (примеры DAG\_A и DAG\_B) характеризуется относительно небольшим масштабом графа работ, небольшим числом процессоров, полнотой графа связности процессоров и одинаковыми задержками между любыми двумя процессорами.
    \item Второй класс (примеры DAG\_C и DAG\_D) характеризуется относительно большим масштабом графа работ, большим числом процессоров
\end{enumerate}
\begin{table}[!htbp]
    \caption{Сравнение примеров из классов данных}
    \begin{tabular}{c|c|c|c|c}
                            & \multicolumn{4}{c}{Примеры входных данных}                            \\
        \hline
        Критерии            & DAG\_A                                     & DAG\_B & DAG\_C & DAG\_D \\
        \hline
        Масштаб графа работ & \makecell{45 вершин;                                                  \\75 ребер}                        & \makecell{1121 вершина;\\6229 ребер} & \makecell{197494 вершин;\\719389 ребер} & \makecell{1823309 вершин;\\6172920 ребер} \\
        \hline
        \makecell{Разброс                                                                           \\длительностей работ}            & 1-10                                       & 1-10   & \makecell{все работы\\одной длины} & \makecell{все работы\\одной длины} \\
        \hline
        \makecell{Связность                                                                         \\графа работ}                  & 1.66                                       & 5.55   & 3.64                   & 3.83                   \\
        \hline
        \makecell{Количество                                                                        \\процессоров}                 & 2                                          & 10     & 256                    & 4096                   \\
        \hline
        \makecell{Полный граф                                                                       \\связности\\процессоров}      & да                                         & да     & нет                    & да                     \\
        \hline
        \makecell{Одинаковые задержки                                                               \\на передачу данных} & да                                         & да     & да                     & нет                    \\
    \end{tabular}

\end{table}

\end{document}