Классическая задача построения расписания хорошо изучена и досканально описана в \cite{Coffman}. Поскольку данная задача принадлежит к классу NP-трудных, не существует алгоритма, который за полиномиальное время даст точный ответ, но существуют алгоритмы, которые дают приближенные результаты. Большинство таких алгоритмов резделяются на две категории: \textit{конструктивные} и \textit{итерационные}. Из основных примеров можно выделить (большинство из них упомянуты в \cite{Kostenko_2017}):
\begin{itemize}
    \item Конструктивные алгоритмы
    \begin{enumerate}
        \item Алгоритмы, основанные на поиске максимального потока в сети
        \item Алгоритмы, основанные на методах динамического программирования
        \item Алгоитмы, основанные на методе ветвей и границ
        \item Жадные алгоритмы
        \item Жадные алгоритмы с процедурой ограниченного перебора
    \end{enumerate}
    \item Итерационные алгоритмы
    \begin{enumerate}
        \item Генетические алгоритмы
        \item Дифференциальная эволюция
        \item Алгоритм имитации отжига
        \item Алгоритм муравьиных колоний
    \end{enumerate}
\end{itemize}
\par
Конструктивные алгоритмы работают, строя и дополняя частичные расписания до тех пор, пока все работы не будут размещены. Итерационные же алгоритмы строят приближения расписания и оптимизируют их.
\par
В данной работе рассматриваются жадные алгоритмы с процедурой ограниченного перебора. Особенностью таких алгоритмов является баланс между двумя процессами построения расписания. Жадные стратегии строят расписание быстро, однако очень быстро могут зайти в тупик при построении расписания. В таком случае, если расписание строится си сильным отклонением от оптимального, процедура ограниченного перебора корректирует его.