\begin{frame}
    \frametitle{Дополнительные обозначения}
    \begin{enumerate}
        \item $D= \left( d_1, d_2, \dots, d_l \right)$, где $l$ - количество вершин, доступных для добавления.
        \item $K$ - вектор длин критических путей от "головной" вершины до каждой вершины графа.
        \item $\left( s_i, p_i \right)$ - достаточное количество информации для размещения работы в расписании. Установка соотношения между работой $t$ и парой $\left( s_i, p_i \right)$ и есть построение расписания
    \end{enumerate}
    \hrule
    \vspace{2pt}
    Жадные критерии
    \begin{enumerate}
        \item $GR1$ - критерий, используемый в выборе работы на постановку
        \item $GR2$ - критерий, используемый в выборе места постановки работы
    \end{enumerate}
    \hrule
    \vspace{2pt}
    Процедуры ограниченного перебора
    \begin{enumerate}
        \item $H1$ - процедура перебора для создания места для постановки работы
        \item $H2$ - процедура перебора для приближения времени старта работы к длине критического пути до нее
    \end{enumerate}
\end{frame}

% \subsection{Схема}
% \begin{frame}
%     \frametitle{Общая схема алгоритма}
%     {\tiny
%         \ctikzfig{algo_schema}
%     }
% \end{frame}

\subsection{Предподсчет}

% \begin{frame}
%     \frametitle{Блок-схема предподсчета}
%     {\tiny
%         \ctikzfig{precalc}
%     }
% \end{frame}

\begin{frame}
    \frametitle{Предподсчет}
    \begin{columns}
        \begin{column}{0.6\textwidth}
            \begin{enumerate}
                \item Формируется множество $D$
                \item Вычисляется вектор $K$. В случае, если такой вершины нет - создается фиктивная вершина с нулевой длительностью. Вектор $k$ заполняется при помощи алгоритма Дейкстры.
            \end{enumerate}
        \end{column}
        \begin{column}{0.4\textwidth}
            \ctikzfig{fictive_nodes}
        \end{column}
    \end{columns}
\end{frame}


\subsection{Пробное размещение работы}
\begin{frame}
    \frametitle{Блок-схема пробного размещения работы}
    {\tiny
        \ctikzfig{choosing}
    }
\end{frame}

\begin{frame}
    \frametitle{Жадный критерий выбора размещения}
    Из множества $\color{red}D$ выбирается работу по критерию $GC1$ максимальности количества потомков у вершины.
    \ctikzfig{max_children}
\end{frame}

\begin{frame}
    \frametitle{Пробное размещение работы}
    Пробное размещение работы производится с учетом  жадного и дополнительных критериев. \\
    Жадный критерий $GC2$ - скорейшее завершение работы в расписании. \\
    Способы выбора места:
    \begin{enumerate}
        \item Подсчет усредненного взвешенного показателя среди критериев \\
              \begin{gather*}
                  crit_{CR} = C_1 \cdot GC2 + C_2 \cdot CR + C_3 \cdot CR2 \\
                  crit_{BF} = C_1 \cdot GC2 + C_2 \cdot BF
              \end{gather*}
              где $C_1$, $C_2$ и $C_3$ - параметры алгоритма. Работа размещается на место с наибольшим значением параметра $crit$.
        \item Допускная система выбора
    \end{enumerate}
\end{frame}

\begin{frame}
    \frametitle{Допускная система выбора}
    \begin{enumerate}
        \item Список мест размещения работ ранжируется по $GC2$, после чего отсекаются верхние $n\%$ работ, где $n$ - параметр алгоритма
        \item Такие же действия повторяются для каждого дополнительного критерия
        \item В конечном списке выбрать место по жадному критерию
    \end{enumerate}

\end{frame}


\subsection{Ограниченный перебор}
\begin{frame}
    \frametitle{Процедура ограниченного перебора}
    \begin{enumerate}
        \item После неудачной пробной постановки работы в расписание алгоритм создает набор $T=\left( t_1,t_2,\dots,t_s \right)$, состоящий из $s$ последних добавленных работ ($s$ – параметр алгоритма).
        \item Процедурой полного перебора пробуются различные расписания до тех пор, пока не получится расписание, удовлетворяющее заданным критериям.
    \end{enumerate}
\end{frame}

\subsection{Корректировка критичности пути}
\begin{frame}
    \frametitle{Блок схема корректировки расписания}
    {\tiny
        \ctikzfig{criticalness}
    }
\end{frame}
