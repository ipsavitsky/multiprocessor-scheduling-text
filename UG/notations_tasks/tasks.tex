\documentclass[a4paper,12pt]{extarticle}
\usepackage[utf8]{inputenc}
\usepackage[T2A]{fontenc}
\usepackage[russian]{babel}
\usepackage{titling}
\usepackage{amsmath}

\setlength{\droptitle}{-15em}

\title{Системы счисления}
\author{Тетерин Дмитрий Юрьевич и Савицкий Илья Павлович}
\date{Июль 2022}


\begin{document}
    \maketitle
    \hrule
    \begin{enumerate}
        \item Первести:
        \begin{itemize}
            \item $1231_{10}$ в шестнадцатеричную
            \item $255_{10}$ в двоичную
            \item $10010101_2$ в десятичную
            \item $AA452_{16}$ в двоичную, а потом в восьмеричную при помощи быстрых алгоритмов
        \end{itemize}
        \item Доказать, что
        \begin{itemize}
            \item если у числа сумма цифр в пятиричной системе счисления делится на 4, то само число делится на 4.
            \item если у числа сумма цифр в 17-ричной системе счисления делится на 4, то само число делится на 4.
        \end{itemize}
        \item Доказать, что в системе счисления с основанием 17 все степени числа будут оканчиваться на разные цифры.
        \item Доказать, что существуют такие записи чисел $a$ и $b$, что
        \begin{gather*}
            a_3 - b_3 = a_2 + b_2
        \end{gather*}
        
    \end{enumerate}
\end{document}