\documentclass[hyperref=unicode, aspectratio=169]{beamer}
\usepackage[utf8]{inputenc}
\usepackage[T2A]{fontenc}
\usepackage[russian]{babel}
% \usepackage{NotesTeX_rus}

\uselanguage{russian}
\languagepath{russian}

\beamertemplatenavigationsymbolsempty

% \deftranslation[to=russian]{example}{пример}
\deftranslation[to=russian]{Example}{Пример}
% \deftranslation[to=russian]{definition}{определение}
\deftranslation[to=russian]{Definition}{Определение}

\title{Системы счисления и двоичная запись }
\author[]{Савицкий Илья Павлович и Тетерин Дмитрий Юрьевич}
\date[]{Июль 2022}

\usetheme{Madrid}

\begin{document}

\maketitle

\begin{frame}{Определения}
    \begin{definition}{\textbf{Система счисления}}
        - знаковая система, в которой приняты определенные правила записи чисел.
    \end{definition}
    \begin{definition}{\textbf{Цифры}}
        - знаки, при помощи которых записываются числа.
    \end{definition}
    \begin{definition}{\textbf{Алфавит}}
        системы счисления - совокупность цифр.
    \end{definition}
\end{frame}

\begin{frame}{Непозиционные системы счисления}
    \begin{itemize}
        \item Унарная \\
              Самая первая из всех созданных. Алфавит: | \\
              $5_{10} = |||||$
        \item Римская \\
              Самая известная на данный момент. Значение символа диктуется не абсолютным, а относительным положением в числе.
        \item \ldots и много много других архаизмов, которые показали свою нежизнеспособность тем, что с ними сложно проводить арифметику
    \end{itemize}
\end{frame}

\begin{frame}{Знакомые нам системы счисления}
    \begin{example}
        Десятичная система счисления состоит из цифр:
        \begin{gather*}
            0, 1, 2, 3, 4, 5, 6, 7, 8, 9
        \end{gather*}
        Это её алфавит.
    \end{example}
\end{frame}

\begin{frame}{Как работает алфавит?}
    \begin{example}
        Как составить \textit{любое} число при помощи алфавита?
        \begin{gather*}
            548 = 5 \cdot 100 + 4 \cdot 10 + 8
        \end{gather*}
        или
        \begin{gather*}
            548 = 5 \cdot 10^2 + 4 \cdot 10^1 + 8 \cdot 10^0
        \end{gather*}
    \end{example}
    \begin{definition}
        10 в, в данном случае, называется \textbf{основанием} системы счисления. (поэтому система и называется десятичной)
    \end{definition}
\end{frame}

\begin{frame}{Как мы считаем?}
    Алфавит: $0, 1, 2, 3, 4, 5, 6, 7, 8, 9$
    \begin{gather*}
        0      \\
        1      \\
        \ldots \\
        9      \\
        10     \\
        11     \\
        \ldots \\
        19 \\
        \ldots \\
        99 \\
        100
    \end{gather*}
\end{frame}

\begin{frame}{Двоичная система счисления}
    \begin{example}
        Двоичная система счисления состоит из цифр:
        \begin{gather*}
            0, 1
        \end{gather*}
        Идея состоит в том, что специальные электрические элементы внутри компьютеров, называемые транзисторами, могут находиться только в двух состояниях: HIGH и LOW.
    \end{example}
\end{frame}

\begin{frame}{Двоичный счет}
    Алфавит: $0, 1$
    \begin{align*}
        0  \qquad & 0    \\
        1  \qquad & 1    \\
        2  \qquad & 10   \\
        3  \qquad & 11   \\
        4  \qquad & 100  \\
        5  \qquad & 101  \\
        6  \qquad & 110  \\
        7  \qquad & 111  \\
        8  \qquad & 1000 \\
        9  \qquad & 1001 \\
        10 \qquad & 1010 \\
    \end{align*}
\end{frame}

\begin{frame}{Интересная особенность}
    Все арифметические правила \textit{сохраняются} в двоичной системе счисления.
    \begin{example}
        \begin{gather*}
            10011_2 = 1 \cdot 2^4 + 0 \cdot 2^3 + 0 \cdot 2^2 + 1 \cdot 2^1 + 1 \cdot 2^0
        \end{gather*}
    \end{example}
\end{frame}

\begin{frame}{Сложение и умножение в ДСС}
    \begin{columns}
        \begin{column}{0.5\textwidth}
            \begin{table}
                \begin{tabular}{c | c | c}
                    + & 0 & 1  \\
                    \hline
                    0 & 0 & 1  \\
                    \hline
                    1 & 1 & 10 \\
                \end{tabular}
            \end{table}
        \end{column}
        \begin{column}{0.5\textwidth}
            \begin{table}
                \begin{tabular}{c | c | c}
                    $\times$ & 0 & 1 \\
                    \hline
                    0        & 0 & 0 \\
                    \hline
                    1        & 0 & 1 \\
                \end{tabular}
            \end{table}
        \end{column}
    \end{columns}
    \begin{example}
        \begin{columns}
            \begin{column}{0.5\textwidth}
                \begin{gather*}
                    \begin{array}{r}
                        -
                        \begin{array}{r}
                            10000011 \\
                            10000001 \\
                        \end{array} \\
                        \hline
                        \begin{array}{r}
                            10
                        \end{array}
                    \end{array}
                \end{gather*}
            \end{column}
            % \begin{column}{0.5\textwidth}
            %     \begin{gather*}
            %         \begin{array}{r}
            %             +
            %             \begin{array}{r}
            %                 3556\overset{1}{2} \\
            %                 399\\
            %             \end{array} \\
            %             \hline
            %             \begin{array}{r}
            %                 35961
            %             \end{array}
            %         \end{array}    
            %     \end{gather*}
            % \end{column}
        \end{columns}
    \end{example}
\end{frame}

\begin{frame}{Основная формула}
    \begin{definition}
        \textbf{Развернутая форма записи}
        \begin{gather*}
            A_q = \pm \left( a_{n-1} \cdot q^{n-1} + a_{n-2} \cdot q^{n-2} + \ldots + a_{0} \cdot q^{0} + \ldots + a_{-1} \cdot q^{-1} + \ldots + a_{-m} \cdot q^{-m} \right)
        \end{gather*}
        , где \\
        $A$ - число \\
        $q$ - основание системы счисления \\
        $a_i$ - цифры, принадлежащие алфавиту данной системы \\
        $n$ - количество целых разрядов числа \\
        $m$ - количество дробных разрядов числа \\
        $q^i$ - \glqq вес\grqq разряда\\
    \end{definition}
\end{frame}

% \begin{frame}{Упражнение 1}
%     Запишите в развернутой форме число $123,456$ в десятичной системе счисления
%     \vfill
% \end{frame}

% \begin{frame}{Упражнение 2}
%     Запишите в развернутой форме число $100011$ в двоичной системе счисления
%     \vfill
% \end{frame}

\begin{frame}{Перевод системы счисления в десятичную}
    Очевидно, что если записать число в развернутой форме, но считать все в привычной нам десятичной форме получится исходное число.
    \begin{example}
        \begin{align*}
            10011_2 & = 1 \cdot 2^4 + 0 \cdot 2^3 + 0 \cdot 2^2 + 1 \cdot 2^1 + 1 \cdot 2^0 \\
                    & = 1 \cdot 16 + 0 \cdot 8 + 0 \cdot 4 + 1 \cdot 2 + 1 \cdot 1          \\
                    & = 16 + 2 + 1                                                          \\
                    & = 19_{10}
        \end{align*}
    \end{example}
\end{frame}

% \begin{frame}{Перевод десятичной системы счисления в двоичную}
%     \begin{align*}
%         \frac{a_{n-1} \cdot 2^{n-1} + a_{n-2} \cdot 2^{n-2} + \ldots + a_1 \cdot 2^1 + a_0}{2} & = a_{n-1} \cdot 2^{n-2} + \ldots + a_1 \quad \text{остаток} \; a_0 \\
%         \frac{a_{n-1} \cdot 2^{n-1} + a_{n-2} \cdot 2^{n-2} + \ldots + a_1}{2}                 & = a_{n-1} \cdot 2^{n-2} + \ldots + a_2 \quad \text{остаток} \; a_1 \\
%         \frac{a_{n-1} \cdot 2^{n-1} + a_{n-2} \cdot 2^{n-2} + \ldots + a_2}{2}                 & = a_{n-1} \cdot 2^{n-2} + \ldots + a_3 \quad \text{остаток} \; a_2 \\
%                                                                                                & \ldots
%     \end{align*}
%     На $n$-м шаге получаем число $\overline{a_0a_1a_2\ldots a_{n-1}}$
% \end{frame}

\begin{frame}{Перевод десятичной системы счисления в двоичную}
    \begin{example}
        \begin{table}
            \begin{tabular}{| c | c | c | c | c | c | c | c | c |}
                \hline
                363 & 181 & 90 & 45 & 22 & 11 & 5 & 2 & 1 \\
                \hline
                1   & 1   & 0  & 1  & 0  & 1  & 1 & 0 & 1 \\
                \hline
            \end{tabular}
        \end{table}
        Итого, получаем число $363_{10} = 101101011_{2}$
    \end{example}
    \begin{example}
        \begin{table}
            \begin{tabular}{|c | c | c | c | c | c | c | c | c |}
                \hline
                314 & 157 & 78 & 39 & 19 & 9 & 4 & 2 & 1 \\
                \hline
                0   & 1   & 0  & 1  & 1  & 1 & 0 & 0 & 1 \\
                \hline
            \end{tabular}
        \end{table}
        Итого, получаем число $314_{10} = 100111010_{2}$
    \end{example}
\end{frame}

\begin{frame}{Шестнадцатеричная система счисления}
    На самом деле, можно придумать цифры после 9, но вместо этого, для оснований системы выше 10 принято использовать латинские буквы.
    \begin{definition}
        Шестнадцатеричная система счисления - система счсления с основанием 16, где
        \begin{align*}
            A & = 10 \\
            B & = 11 \\
            C & = 12 \\
            D & = 13 \\
            E & = 14 \\
            F & = 15 \\
        \end{align*}
    \end{definition}
\end{frame}

\begin{frame}{Fast bin2hex и fast hex2bin}
    В системах счисления с основанием, равным степени двойки наблюдается интересная особенность: числа можно переводить посимвольно!
    \begin{example}
        \begin{gather*}
            \underbrace{1}_{1} \underbrace{0001}_{1} \underbrace{0101}_{5} \underbrace{1010}_{A} \underbrace{1001}_{9} \underbrace{0010_{2}}_{2} \\
            100010101101010010010_2 = 115A92_{16}
        \end{gather*}
    \end{example}
    \begin{example}
        \begin{gather*}
            \underbrace{7}_{0111} \underbrace{3}_{0011} \underbrace{B_{16}}_{1011} \\
            73B_{16} = 11100111011_{2}
        \end{gather*}
    \end{example}
\end{frame}

\begin{frame}{Двоично-десятичная система счисления}
    Двоично-десятичная система счисления (BCD) повторяет идею fast bin2hex и fast hex2bin только для пары десятичной и двоичной систем счисления.
    \begin{example}
        \begin{gather*}
            \underbrace{3}_{0011}\underbrace{1}_{0001}\underbrace{1_{10}}_{0001} \\
            311_{10} = 1100010001_{BCD}
        \end{gather*}
    \end{example}
\end{frame}

\begin{frame}{Симметричная троичная система счисления}
    Алфавит: $\overline{1}, 0, 1$ \\
    Сохраняется схема с позиционными аналогами, но цифра $\overline{1}$ теперь обозначает $-1$.
    \begin{table}
        \begin{tabular}{| c | c | c | c | c | c | c | c | c | c | c | c |}
            \hline
            Десятичная система             & $-5$             & $-4$            & $-3$            & $-2$            & $-1$           & $0$ & $1$ & $2$             & $3$  & $4$  & $5$              \\
            Троичная несимметричная        & $-12$            & $-11$           & $-10$           & $-2$            & $-1$           & $0$ & $1$ & $2$             & $10$ & $11$ & $12$             \\
            \textbf{Троичная симметричная} & $\overline{1}11$ & $\overline{11}$ & $\overline{1}0$ & $\overline{1}1$ & $\overline{1}$ & $0$ & $1$ & $1\overline{1}$ & $10$ & $11$ & $1\overline{11}$ \\
            \hline
        \end{tabular}
    \end{table}
    \begin{example}
        \begin{align*}
            1\overline{1}01\overline{1} & = 1 \cdot 3^4 + \overline{1} \cdot 3^3 + 0 \cdot 3^2 + 1 \cdot 3^1 + \overline{1} \cdot 3^0 \\
                                        & = 81 - 27 + 3 - 1                                                                           \\
                                        & = 56
        \end{align*}
    \end{example}
\end{frame}

\end{document}
