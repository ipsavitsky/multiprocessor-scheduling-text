\documentclass[hyperref=unicode, aspectratio=169]{beamer}
\usepackage[utf8]{inputenc}
\usepackage[T2A]{fontenc}
\usepackage[russian]{babel}
\usepackage{minted}

\uselanguage{russian}
\languagepath{russian}

\deftranslation[to=russian]{Example}{Повтори сам}
\deftranslation[to=russian]{Definition}{Определение}

\title{Введение в Python}
\author[]{Савицкий Илья Павлович и Тетерин Дмитрий Юрьевич}
\date[]{Июль 2022}

\usetheme{Madrid}
% \useoutertheme{infolines}
% \setbeamersize{text margin left=1em,text margin right=1em}

\setminted[python]{linenos=true}

\begin{document}
\maketitle

\begin{frame}{Ставим Python}
    \begin{enumerate}
        \item Заходим в учетки под своими логинами и паролями (если не помните - спросите у нас)
        \item Заходим в Microsoft Store, в поиске вводим python
        \item Скачиваем то что будет первое в результате
        \item Profit :)
    \end{enumerate}
\end{frame}

\begin{frame}[fragile]{Первая программа на Python}
    Для начала посмотрим пример самой простой функции - \mintinline{python}{print()}. Она не делает ничего, кроме вывода на экран того, что вы подали ей на вход.
    \begin{example}
        \begin{minted}{python}
print("Hello, world!")
        \end{minted}    
    \end{example}
    Функции, которые стоят последовательно выполняются одна за другой.
    \begin{example}
        \begin{minted}{python}
print("Привет,")
print("мир!")
        \end{minted}    
    \end{example}
\end{frame}

\begin{frame}[fragile]{Переменные}
    Иногда в программе надо что-то хранить, например, числа. У каждой переменной должно быть имя (ставится слева от равно) и значение (ставится справа от равно)
    \begin{example}
        \begin{minted}{python}
a = 5
b = 7.6
print(b)
print(a)
b = a
print(b)
        \end{minted}    
    \end{example}
    
\end{frame}

\begin{frame}[fragile]{Математические операторы}
    Математические операции в Python \textit{просто работают}, что с целыми, что с дробными числами
    \begin{example}
        \begin{minted}{python}
a = 5
b = 7
print(a + b)
print(a - b)
print(a * b)
print(a / b)
print(a % b)
print(a ** b)
        \end{minted}    
    \end{example}
\end{frame}

\begin{frame}[fragile]{Строковый тип}
    Строки в Python работают точно так же:
    \begin{example}
        \begin{minted}{python}
a = "abc"
b = 'rds'
print(a + b)
        \end{minted}    
    \end{example}
    
    Таким образом, уже бывший пример
    \begin{minted}{python}
print("Hello world!")
    \end{minted}
    абсолютно идентичен следующему:
    \begin{minted}{python}
a = "Hello world"
print(a)
    \end{minted}
\end{frame}

\begin{frame}[fragile]{Ввод, но что-то нечисто}
    За ввод в Python отвечает функция \mintinline{python}{input()}. Давайте попробуем с ее помощью ввести число и провести с ним какие-то манипуляции.
    \begin{example}
        \begin{minted}{python}
a = input()
print(a + 10)
        \end{minted}
    \end{example}
    Получаем ошибку:
    \begin{minted}{text}
Traceback (most recent call last):
  File "<stdin>", line 1, in <module>
TypeError: can only concatenate str (not "int") to str
    \end{minted}
    Которая, если перевести ее на русский, говорит что нельзя сложить число и строку. Все потому, что \mintinline{python}{input()} \textit{всегда} вводит с клавиатуры строку, то есть нам с клавиатуры ввелось не число, а строка \mintinline{python}{'5'}
\end{frame}

\begin{frame}[fragile]{Ввод, но все теперь чисто}
    Чтобы превратить строку в число воспользуемся простой функцией \mintinline{python}{int()}. Тогда \mintinline{python}{a} присвоится значение именно соответствующего числа.
    \begin{example}
        \begin{minted}{python}
a = int(input())
print(a + 10)
        \end{minted}    
    \end{example}
    Попробуйте ввести как число, так и что-то, что в число не приводится, например строку \mintinline{python}{"Привет"}. \\
    Чтобы привести строку к дробному числу можно использовать \mintinline{python}{float()}
\end{frame}

\begin{frame}[fragile]{Списки и слайсы}
    Очень часто в программировании приходится хранить \textit{набор} каких-то значение, и чтобы не хранить каждое значение в отдельной переменной, в Python существуют \textbf{списки}. Каждый элемент в таком списке пронумерован \textbf{с нуля}. Отрицательные индексы считаются с конца. \mintinline{python}{array[-1]} вернет последний элемент. Слайсы принимают два индекса - с какого по какого элемента взять в слайсе. Дополнительно можно указать третий элемент - шаг.
    \begin{example}
        \begin{minted}{python}
a = [4, 3, 2, 1, "два", 2.5, "ТРИ"]
print(a)
print(a[1])
print(a[-2])
print(a[2:5])
print(a[:2])
print(a[3::2])
        \end{minted}
    \end{example}
    
\end{frame}

\begin{frame}[fragile]{Разные условия в Python}
    \begin{definition}
        \textbf{Булевым} типом данных называет тип данных, который может принимать только два значения. В Python эти значения - \mintinline{python}{True} и \mintinline{python}{False}
    \end{definition}
    \begin{enumerate}
        \item Сравнение на равенство (\mintinline{python}{==}) или неравенство (\mintinline{python}{!=})
        \begin{example}
            \begin{minted}{python}
print(5 == 5)
print(5 == 6)
print(5 != 6)
print(6 != 6)
            \end{minted}
        \end{example}
        \item Больше (\mintinline{python}{>})/больше или равно (\mintinline{python}{>=}), меньше (\mintinline{python}{<})/меньше или равно (\mintinline{python}{<=})
    \end{enumerate}
\end{frame}

\begin{frame}[fragile]{Оператор if}
    Писать линейные программы скучно - одна из возможностей сменить привычный ход программы - оператор \mintinline{python}{if}
    \begin{example}
        Пример программы, проверяющей четность числа
        \begin{minted}{python}
a = int(input())
if a % 2 == 0:
    print("Четное")
else:
    print("Нечетное")
        \end{minted}    
    \end{example}
\end{frame}

\begin{frame}[fragile]{Цикл for}
    Оператор \mintinline{python}{for} повторяет какую-то часть кода, при этом постоянно сменяя какую-то переменную. Понятнее становится из следующего кода: 
    \begin{example}
        \begin{minted}{python}
a = [2, 4, 6, 8, 10, 12]
for i in a:
    print(i)
    print(i**2)
        \end{minted}
    \end{example}
\end{frame}

\begin{frame}[fragile]{Цикл for}
    Чтобы быстро и на месте сгенерировать список нужной длины можно воспользоваться специальной функцией \mintinline{python}{range()}. Несколько примеров 
    \begin{example}
        \begin{minted}{python}
for i in range(10):
    print(i)
        \end{minted}
    \end{example}
\end{frame}

\begin{frame}[fragile]{Цикл while}
    \begin{example}
        \begin{minted}{python}
a = 123
while a > 0:
    print(a)
    a = a // 2
        \end{minted}
    \end{example}
    \begin{example}
        \begin{minted}{python}
while True:
    a = int(input())
    if a == 0:
        break
    ...
        \end{minted}
    \end{example}
\end{frame}

\begin{frame}[fragile]{Пора поделать что-то самим}
    \begin{enumerate}
        \item Открываем сайт ejudge.school.msu.ru
        \item Выбираем соответствующее соревнование
        \item Заходим под логином/паролем с фото
        \item Меняем свои логины/пароли
        \item Решаем задачки
        \item Profit :)
    \end{enumerate}
\end{frame}

\end{document}