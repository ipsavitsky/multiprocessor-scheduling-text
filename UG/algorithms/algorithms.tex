\documentclass[hyperref=unicode, aspectratio=169]{beamer}
\usepackage[utf8]{inputenc}
\usepackage[T2A]{fontenc}
\usepackage[russian]{babel}
\usepackage{minted}

\uselanguage{russian}
\languagepath{russian}

\beamertemplatenavigationsymbolsempty

\deftranslation[to=russian]{Example}{Пример}
\deftranslation[to=russian]{Definition}{Определение}

\title{Разбор контеста}
\author[]{Савицкий Илья Павлович и Тетерин Дмитрий Юрьевич}
\date[]{Июль 2022}

\usetheme{Berlin}

\begin{document}

\maketitle
\section{Задача А}
\begin{frame}[fragile]{Правильное решение}
    \begin{minted}{python}
a = int(input())
b = int(input())

print(a + b)
    \end{minted}    
\end{frame}


\section{Задача B}

\begin{frame}[fragile]{Распространенная ошибка 1}
    \begin{minted}{python}
a = int(input())

if a % 2 = 0:
   print("YES")
else:
   print("NO")
    \end{minted}
\end{frame}

\begin{frame}[fragile]{Правильное решение}
    \begin{minted}{python}
a = int(input())

if a % 2 == 0:
   print("YES")
else:
   print("NO")
    \end{minted}
\end{frame}


\section{Задача C}

\begin{frame}[fragile]{Правильное решение 1}
    \begin{minted}{python}
a=int(input())

if a%3==0:
    if a%6==0:   
        print("NO")
    else:
        print("YES")
else:
    print("NO")
    \end{minted}
\end{frame}

\begin{frame}[fragile]{Сложные условия}
    Сложные условия позволяют избежать вложенных \mintinline{python}{if}-ов и сильно упрощают код.
    \begin{minted}{python}
if <условие1> and <условие2>:
    ...
if <условие1> or <условие2>:
    ...
if not <условие1>:
    ...
    \end{minted}
\end{frame}

\begin{frame}[fragile]{Правильное решение 2}
    \begin{minted}{python}
a = int(input())

if a % 3 == 0 and a % 6 != 0:
   print("YES")
else:
   print("NO")
    \end{minted}
\end{frame}


\section{Задача D}

\begin{frame}[fragile]{Правильное решение 1}
    \begin{minted}{python}
a= int(input())
if a>0:
    while a != 0:
        print(a%10, end="")
        a= a//10
elif a<0:
    a=-a
    print('-', end="")
    while a != 0:
        print(a%10, end="")
        a= a//10
elif a==0:
    print(0)
    \end{minted}
\end{frame}

\begin{frame}[fragile]{Правильное решение 2}
    \begin{minted}{python}
a = int(input())
if a > 0:
    a = str(a)
    b = a[::-1]
    print(b)
else:
    b = abs(a)
    b = str(b)
    c = b[::-1]
    c = int(c)
    c = c - c - c
    print(c)
    \end{minted}
\end{frame}

\section{Задача E}

\begin{frame}[fragile]{Правильное решение 1}
    \begin{minted}{python}
a = int(input())
m = a
while a != 0:
    if a > m:
        m = a
    a = int(input())
print(m)
    \end{minted}
\end{frame}

\section{Задача F}

\begin{frame}[fragile]{Правильное решение 1}
    \begin{minted}{python}
a = int(input())
b = int(input())
if b!=0:
    m = max(a, b)
    m2 = min(a, b)
else:
    m = a
    m2 = a
while a != 0 and b!= 0:
    if a > m:
        m2 = m
        m = a
    a = int(input())
print(m)
print(m2)
    \end{minted}
\end{frame}

\begin{frame}[fragile]{Правильное решение 1}
    \begin{minted}{python}
n = int(input())
temp = n
rev = 0
while(n > 0):
    dig = n % 10
    rev = rev * 10 + dig
    n = n // 10
if(temp == rev):
    print("YES")
else:
    print("NO")
    \end{minted}
\end{frame}

\end{document}