\begin{frame}
    \frametitle{Постановка задачи}
    \begin{columns}
        \begin{column}{0.60\textwidth}
            Дано:
            \begin{enumerate}
                \item Ориентированный граф работ $G$ без циклов, в котором дуги - зависимости по данным, а вершины - задания. Вершин $n$, дуг $m$
                \item Вычислительная система, состоящая из $p$ различных процессоров
                \item Матрица $C_{ij}$ длительности выполнения работ на процессорах, $i=1 \dots n, j=1 \dots p$. Каждая строка этой матрицы - длины выполнения $n$-й задачи на $p$ процессорах. 
                \item Матрица $D_{kl}$ передач данных между процессорами, $k=1 \dots p, l = 1 \dots p, D_{kk} = 0$. $D_{ij}$-й элемент этой матрицы - время передачи данных между процессорами $i$ и $j$.
            \end{enumerate}
        \end{column}
        \begin{column}{0.40\textwidth}
            \begin{figure}
                \ctikzfig{graph_schema}
                \captionsetup{labelformat=empty}
                \caption{Граф потока данных}
            \end{figure}
        \end{column}
    \end{columns}
\end{frame}

\begin{frame}
    \frametitle{Расписание}
    Расписание программы определено, если определены
    \begin{enumerate}
        \item Множества процессоров и работ;
        \item привязка;
        \item порядок;
    \end{enumerate}
    \par
    Привязка - всюду определенная на множестве работ функция, которая задает распределение работ по процессорам
    \par
    Порядок задает ограничения на последовательность выполнения работ и является отношением частичного порядка, удовлетворяющим условиям ацикличности и транзитивности. Отношение порядка на множестве работ, распределенных на \\один процессор, является отношением полного порядка.
\end{frame}

\begin{frame}
    \frametitle{Графическая форма представления расписания}
    \begin{figure}
        \small
        \ctikzfig{figures/schedule-graphical-form}
        \captionsetup{labelformat=empty}
        \caption{\small Графическая форма представления расписания}
    \end{figure}
    Графическая форма представления расписания $\Leftrightarrow$ Временная диаграмма
\end{frame}

\begin{frame}
    \frametitle{Постановка задачи}
    \begin{columns}
        \begin{column}{0.7\textwidth}
            Требуется:
            \begin{enumerate}
                \item Построить расписание $HP$, то есть для $i$-й работы определить время начала ее выполнения $s_i$ и процессор $p_i$ на которм она будет выполняться
                \item Минимизируемый критерий: время завершения выполнения расписания
            \end{enumerate}
        \end{column}
        \begin{column}{0.3\textwidth}
            \begin{figure}
                \tiny
                \ctikzfig{figures/schedule-time-diagram}
                \captionsetup{labelformat=empty}
                \caption{\small Представление расписания в виде временной диаграммы}
            \end{figure}
        \end{column}
    \end{columns}
\end{frame}

\begin{frame}
    \frametitle{Модель расписания}
    Множество корректных расписаний $HP$ задается набором ограничений:
    \begin{itemize}
        \item В расписании не допустимы прерывания
        \item Интервалы выполнения работ на процессоре не пересекаются
        \item Каждая работа назначена на процессор
        \item Любую работу обслуживает один процессор
        \item Частичный порядок, заданный графом зависимостей $G$, сохранен в $HP: G \subset G_{HP}^T$, где $G_{HP}^T$ - транзитивное замыкание отношения $G_{HP}$
    \end{itemize}
\end{frame}

\begin{frame}
    \frametitle{Постановки задачи}
    \begin{enumerate}
        \item Задача с однородными процессорами (длительность выполнения работы не зависит от того, на каком процессоре она выполняется) и дополнительными ограничениями на количество передач:
              \begin{itemize}
                  \item $CR = \frac{m_{ip}}{m} \leq 0.4$, где $m_{ip}$ - количество межпроцессорных передач
                        \vspace{5pt}
                  \item $CR2 = \frac{m_{2edg}}{m} \leq 0.05$, где $m_{2edg}$ - количество межпроцессорных передач через третий процессор. Гарантированно, что передача хотя бы через третий процессор всегда есть
              \end{itemize}
        \item Задача с однородными процессорами и дополнительным ограничением сбалансированности распределения работ:
              \begin{itemize}
                  \item $BF = \left\lceil 100 \left( \frac{a_{max} p}{n} - 1 \right) \right\rceil \leq 10$, где $a_{max}$ - наибольшее, по всем процессорам, количество работ на процессоре
              \end{itemize}
        \item Задача с неоднородными процессорами, но без дополнительных \\ограничений на расписание
    \end{enumerate}
\end{frame}
