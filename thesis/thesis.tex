\documentclass[12pt]{article}

\usepackage[utf8]{inputenc}
\usepackage[T2A]{fontenc}
\usepackage[russian]{babel}

\usepackage{csquotes}
\usepackage{amsmath}

\usepackage{makecell}
\usepackage{float}
\usepackage[toc]{appendix}

\usepackage[
backend=biber,
style=numeric,
sorting=ynt
]{biblatex}

\usepackage{tikzit}

\usepackage[a4paper,top=20mm,bottom=20mm,left=30mm,right=10mm]{geometry}

\usepackage{tocloft}
\renewcommand{\cftsecleader}{\cftdotfill{\cftdotsep}}

\newcommand{\intro}[1]{
    \stepcounter{section}
    \section*{\hfillПРИЛОЖЕНИЕ \arabic{section}}
    \begin{center}
        \bf{#1}
    \end{center}
    \markboth{\MakeUppercase{#1}}{}
    \addcontentsline{toc}{section}{Приложение \arabic{section}. #1}
}


% \bibliographystyle{gost2008}
\addbibresource{references.bib}

\input{block-schemas.tikzstyles}

\author{Савицкий Илья Павлович}
\title{Построение многопроцессорного расписания с использованием жадных стратегий и ограниченного перебора}
\date{Москва, 2022}

\begin{document}
\makeatletter
\begin{titlepage}
    \begin{center}
        \includegraphics[width=9cm]{imgs/msulogo.png}\\
        \small
        \centerline{Московский государственный университет имени М.В. Ломоносова}
        \centerline{Факультет вычислительной математики и кибернетики}
        \centerline{Кафедра автоматизации систем вычислительных комплексов}
        \centerline{}
        \Large
        \vfill
        {\@author}\\
        \null
        {\LARGE \bf
        \@title
        }\\
        \null \null
        {\large Курсовая работа}\\
        \null \null
    \end{center}
    \begin{flushright}
        {\bf Научный руководитель:}\\
        Доцент, к.т.н \\Костенко Валерий Алексеевич\\
        ~\\
        % {\bf Научный консультант:}\\
        % к.ф.-м.н.\\Иванов Иван Иванович\\
        \vfill
    \end{flushright}
    \centerline{Москва, 2022}
\end{titlepage}
\setcounter{page}{2}

\newpage
\begin{abstract}
    Построение многопроцессорного расписания это NP-трудная задача. Не существует полиномиального алгоритма  В данной работе приводится один из возможных вариантов решения задачи с дополнительными ограничениями на количество передач или сбалансированость распределения работ на процессорах при помощи алгоритма, сочетающего жадные стратегии и ограниченный перебор.
\end{abstract}
\newpage
\tableofcontents
% ----------------------------------------------------------------------------------------------------------------------
\newpage
\section{Введение}
Классическая задача построения расписания хорошо изучена и досканально описана в \cite{Coffman}. Поскольку данная задача принадлежит к классу NP-hard, не существует алгоритма, который за полиномиальное время даст точный ответ, но существуют алгоритмы, которые дают аппроксимированные результаты. Большинство таких алгоритмов резделябтся на две категории: \textit{конструктивные} и \textit{итерационные}. Из основных примеров можно выделить (большинство из них упомянуты в \cite{Kostenko_2017}):
\begin{itemize}
    \item Конструктивные алгоритмы
    \begin{enumerate}
        \item Алгоритмы, основанные на поиске максимального потока в сети
        \item Алгоритмы, основанные на методах динамического программирования
        \item Алгоитмы, основанные на методе ветвей и границ
        \item Жадные алгоритмы
        \item Жадные алгоритмы с процедурой ограниченного перебора
    \end{enumerate}
    \item Итерационные алгоритмы
    \begin{enumerate}
        \item Генетические алгоритмы
        \item Дифференциальная эволюция
        \item Алгоритм имитации отжига
        \item Алгоритм муравьиных колоний
    \end{enumerate}
\end{itemize}
\par
Конструктивные алгоритмы работают, строя и дополняя частичные расписания до тех пор, пока все работы не будут размещены. Итерационные же алгоритмы строят приближения расписания и оптимизируют их.
\par
В данной работе рассматриваются жадные алгоритмы с процедурой ограниченного перебора. Особенностью таких алгоритмов является баланс между двумя процессами построения расписания. Жадные стратегии строят расписание быстро, однако очень быстро могут зайти в тупик при построении расписания. В таком случае, если расписание строится си сильным отклонением от оптимального, процедура ограниченного перебора корректирует его.
% ----------------------------------------------------------------------------------------------------------------------
\newpage
\section{Цели и задачи курсовой работы}
% TODO: добавить в целом что-то сюда :)
% ----------------------------------------------------------------------------------------------------------------------
\newpage
\section{Постановка задачи}
\subsection*{Дано}
\begin{enumerate}
    \item Ориентированный граф работ $G$ без циклов, в котором дуги - зависимости по данным, а вершины - задания. Вершин $n$, дуг $m$
    \item Вычислительная система, состоящая из $p$ различных процессоров
    \item Матрица $C_{ij}$ длительности выполнения работ на процессорах, $i=1 \dots n, j=1 \dots p$
    \item Матрица $D_{kl}$ передач данных между процессорами, $k=1 \dots p, l = 1 \dots p, D_{kk} = 0$
\end{enumerate}
\subsection*{Определение расписания}
Расписание программы определено, если
\begin{enumerate}
    \item Множества процессор и работ
    \item Привязка - всюду определенная на множестве работ функция, которая задает распределение работ по процессорам
    \item Порядок - заданные ограничения на последовательность выполнения работ и является отношением частичного порядка, удовлетворяющим условиям ацикличности и транзитивности. Отношение порядка на множестве работ, распределенных на один процессор, является отношением полного порядка.
\end{enumerate}
\subsubsection*{Способы представления расписаний}
\begin{enumerate}
    \item Графическая форма представления
    \begin{figure}[H]
        \ctikzfig{schedule-graphical-form}
        \caption{Графическое представление расписания}
    \end{figure}
    В такой форме представления расписания каждой задаче сопоставляется пара из номера процессора и порядкового номера работы на процессоре. 
    \item Временная диаграмма
    \begin{figure}[H]
        \ctikzfig{schedule-time-diagram}
        \caption{Представление расписания в вмде временной диаграммы}
    \end{figure}
    В такой форме представления расписания каждой задаче сопоставляется пара из номера процессора и временис старта задачи на процессоре.
\end{enumerate}
Доказано, что эти формы полностью эквивалентны и, имея одну, возможно построить другую. В предложенном решении расписание строится в виде временной диаграммы, поскольку эта форма легче воспринимается человеком.

\subsection*{Требуется}
\begin{enumerate}
    \item Построить расписание $HP$, то есть для $i$-й работы определить время начала ее выполнения $s_i$ и процессор $p_i$ на которм она будет выполняться
    \item В расписании требуется минимизировать время выполнения набора работ, данных в графе $G$
    \item В задаче так же присутствуют дополнительные ограничения, котрым расписание обязано удовлетворять.
\end{enumerate}
\subsection*{Различные постановки задачи}
\begin{enumerate}
    \item Задача с однородными процессорами (длительность выполнения работы не зависит от того, на каком процессоре она выполняется) и дополнительными ограничениями на количество передач:
          \begin{itemize}
              \item $CR = \frac{m_{ip}}{m} < 0.4$, где $m_{ip}$ - количество передач данных между работами на каждый процессор
              \item $CR2 = \frac{m_{2edg}}{m} < 0.05$, где $m_{2edg}$ - количество дуг, начальный и конечный узлы которых назначены на процессоры, не соединенных напрямую
          \end{itemize}
    \item Задача с однородными процессорами и дополнительным ограничением сбалансированности распределения работ:
          \begin{itemize}
              \item $BF = \lceil 100 \cdot \left( \frac{a_{max} \cdot p}{n} - 1 \right) \rceil < 10$, где $a_{max}$ - наибольшее, по всем процессорам, количество работ на процессоре
          \end{itemize}
    \item Задача с неоднородными процессорами, но без дополнительных ограничений на расписание
\end{enumerate}
% ----------------------------------------------------------------------------------------------------------------------
\newpage
\section{Обзор предметной области}
\subsection{Жадные алгоритмы}
Жадные алгоритмы подразумевают декомпозицию задачи на ряд более простых подзадач \cite{Kostenko_2017}.

\subsection{Жадные алгоритмы с процедурой ограниченного перебора}

% ----------------------------------------------------------------------------------------------------------------------
\newpage
\section{Алгоритм построения расписания}
% \begin{frame}
%     \frametitle{Дополнительные обозначения}
%     \begin{enumerate}
%         \item $D= \left( d_1, d_2, \dots, d_l \right)$, где $l$ - количество вершин, доступных для добавления.
%         \item $\left( s_i, p_i \right)$ - достаточное количество информации для размещения работы в расписании. Установка соотношения между работой $t$ и парой $\left( s_i, p_i \right)$ и есть построение расписания
%     \end{enumerate}
%     \hrule
%     \vspace{2pt}
%     Жадные критерии
%     \begin{enumerate}
%         \item $GC1$ - критерий, используемый в выборе работы на постановку
%         \item $GC2$ - критерий, используемый в выборе места постановки работы
%     \end{enumerate}
%     \hrule
%     \vspace{2pt}
%     EDF-эвристика.
% \end{frame}


\begin{frame}
    \frametitle{Общая схема жадных алгоритмов построения расписания}
    {
        \small
        \begin{tikzpicture}
            \node (start) at (0, 0) [draw, terminal] {Начало};
            \node (decision) at (0, -2) [draw, decision, align=center] {Все работы добавлены \\ в расписание?};
            \node (finish) at (-4, -3) [draw, terminal] {Конец};
            \node (choose_task) at (0, -4.1) [draw, process] {Выбрать следующую работу для постановки};
            \node (choose_proc) at (0, -5.5) [draw, process] {Выбрать процессор для работы};
            \node (add_task) at (6, -5.5) [draw, process] {Поставить работу на процессор};

            \draw[thick, ->] (start) -- (decision);
            \draw[thick, ->] (decision) -- node[right]{Нет} (choose_task);
            \draw[thick, ->] (choose_task) -- (choose_proc);
            \draw[thick, ->] (choose_proc) -- (add_task);
            \draw[thick, ->] (add_task) |- (decision);
            \draw[thick, ->] (decision) -| node[above]{Да} (finish);
        \end{tikzpicture}
    }

\end{frame}

\begin{frame}
    \frametitle{Жадный алгоритм с выбором по числу потомков}
    \begin{enumerate}
        \item Выбор следующей работы на постановку - критерий $GC1$
        \item Выбор процессора для работы
              \begin{itemize}
                  \item Для $CR$ - из изначально заданного распределения
                  \item Для $NO$ - по критерию $GC2$
              \end{itemize}
    \end{enumerate}
    Зададим множество доступных для добавления вершин $D= \left( d_1, d_2, \dots, d_l \right)$, где $l$ - количество вершин, доступных для добавления.
    \begin{columns}
        \begin{column}{0.6\textwidth}
            \textbf{Критерий $GC1$:} \\
            Из множества $\color{red}D$ выбирается работa по критерию $GC1$ максимальности количества потомков у вершины. \\
            \textbf{Критерий $GC2$:} \\
            Работа ставится на процессор, на котором время завершения работы будет минимальным.

        \end{column}
        \begin{column}{0.4\textwidth}
            \ctikzfig{max_children}
        \end{column}
    \end{columns}
\end{frame}

\begin{frame}
    \frametitle{Алгоритм постановки работы на процессор}
    \begin{columns}
        \begin{column}{0.55\textwidth}
            При постановке требуется найти такое минимальное время $t$, чтобы
            \begin{enumerate}
                \item Все передачи данных завершились до $t$;
                \item Существует интервал простоя длительности не меньший времени выполнения работы, начинающийся в $t$.
            \end{enumerate}
        \end{column}
        \begin{column}{0.45\textwidth}
            {
                \tiny
                \ctikzfig{schedule-time-diagram-new-3}
            }
            {
                \tiny
                \ctikzfig{schedule-time-diagram-new-2}
            }
        \end{column}
    \end{columns}
\end{frame}

\begin{frame}
    \frametitle{Жадный алгоритм с фиктивными директивными сроками}
    \begin{enumerate}
        \item Выбор следующей работы на постановку - в порядке возрастания фиктивных директивных сроков;
        \item Выбор процессора для работы:
              \begin{itemize}
                  \item Для $CR$ - из изначально заданного распределения
                  \item Для $NO$ - по критерию $GC2$
              \end{itemize}
    \end{enumerate}
    % Пусть \textbf{длина пути} - сумма всех задержек передач данных и времен выполнения работ на процессорах.
    % Пусть директивный срок всего расписания $d$, а $p_A$ - длина длиннейшего пути от работы $A$ до работы $S$ такой, что у $S$ нет потомков. Тогда директивный срок $d_A$ вершины $A$ равен $d_A - p$.
    % \begin{figure*}
    \begin{columns}
        \begin{column}{0.5\textwidth}
            \ctikzfig{edf}
        \end{column}
        \begin{column}{0.5\textwidth}
            Распространение директивных сроков по графу потока управления.\\Все межпроцессорные передачи равны 2. Только $x_2$ и $x_5$ находятся на разных процессорах.
        \end{column}
    \end{columns}
    % \captionof{figure}{Распространение директивных сроков по графу потока управления.\\Все межпроцессорные передачи равны 2. Только $x_2$ и $x_5$ находятся на разных\\процессорах.}
    % \end{figure*}
\end{frame}



% ----------------------------------------------------------------------------------------------------------------------
\newpage
\section{Программная реализация алгоритма}
\begin{frame}
    \frametitle{Программная реализация}
    Алгоритм реализован на языке \lstinline{C++} с помощюь фреймворка \lstinline{boost}.

    Проект обладает следующей структурой:
    \begin{enumerate}
        \item \lstinline{logging} - функции настройки логирования для проекта. Реализовано на основе \lstinline{Boost::log}
        \item \lstinline{schedule} - модуль для работы с графом входных данных и матрицами $C$ и $D$, подаваемыми на вход. Реализован на основе \lstinline{Boost::graph} и \lstinline{Boost::uBLAS}.
        \item \lstinline{time_schedule} - модуль для работы с временной диаграммой.
        \item \lstinline{main.cpp} - \lstinline{main()} программы. Основной алгоритм реализован тут. Разбор аргументов основан на \lstinline{Boost::program_options}.
        \item \lstinline{Doxyfile} - файл с настройками \lstinline{Doxygen}.
    \end{enumerate}

    Для сборки проекта используется \lstinline{CMake}.
\end{frame}

% ----------------------------------------------------------------------------------------------------------------------
\newpage
\section{Экспериментальное исследование алгоритма}
\begin{frame}
    \frametitle{Точность полученного расписания}
    \begin{columns}
        \begin{column}{0.5\textwidth}
            \begin{figure}
                \includegraphics[width=\textwidth]{imgs/relative_score.png}
            \end{figure}
        \end{column}
        \begin{column}{0.5\textwidth}
            \begin{table}
                \caption*{Наборы исходных данных, используемых в тестировании}
                \begin{tabular}{c | c | c | c}
                    Граф & Вершин & Процессоров & Передач \\
                    \hline
                    1    & 126    & 4           & 716     \\
                    2    & 417    & 8           & 2367    \\
                    3    & 408    & 8           & 8763    \\
                    4    & 296    & 8           & 395     \\
                    5    & 93     & 4           & 92      \\
                \end{tabular}
            \end{table}
            % \begin{enumerate}
            %     \item 126 вершин, 4 процессора, 716 передач данных
            %     \item 417 вершин, 8 процессоров, 2367 передач данных
            %     \item 408 вершин, 8 процессоров, 8763 передач данных
            %     \item 396 вершин, 8 процессоров, 395 передачи данных
            %     \item 93 вершины, 4 процессора, 92 передачи данных
            % \end{enumerate}
        \end{column}
    \end{columns}

    \begin{equation*}
        relative\_score = \frac{\text{время полученного расписания}}{\text{время оптимального расписания}} - 1
    \end{equation*}
\end{frame}

\begin{frame}
    \frametitle{Время выполнения программы}
    \begin{figure}
        \includegraphics[width=0.7\textwidth]{imgs/times.png}
    \end{figure}
\end{frame}

% ----------------------------------------------------------------------------------------------------------------------
\newpage
\section{Заключение}
\input{pages/07conclusion.tex}

% ----------------------------------------------------------------------------------------------------------------------
\newpage
\printbibliography

\appendix

\newpage
\intro{Классы входных данных}
В данных, присланных от Хуавей существует разделение на 2 класса.
\begin{enumerate}
    \item Первый класс (примеры DAG\_A и DAG\_B) характеризуется относительно небольшим масштабом графа работ, небольшим числом процессоров, полнотой графа связности процессоров и одинаковыми задержками между любыми двумя процессорами.
    \item Второй класс (примеры DAG\_C и DAG\_D) характеризуется относительно большим масштабом графа работ, большим числом процессоров
\end{enumerate}
\begin{table}[!htbp]
    \caption{Сравнение примеров из классов данных}
    \begin{tabular}{c|c|c|c|c}
                            & \multicolumn{4}{c}{Примеры входных данных}                            \\
        \hline
        Критерии            & DAG\_A                                     & DAG\_B & DAG\_C & DAG\_D \\
        \hline
        Масштаб графа работ & \makecell{45 вершин;                                                  \\75 ребер}                        & \makecell{1121 вершина;\\6229 ребер} & \makecell{197494 вершин;\\719389 ребер} & \makecell{1823309 вершин;\\6172920 ребер} \\
        \hline
        \makecell{Разброс                                                                           \\длительностей работ}            & 1-10                                       & 1-10   & \makecell{все работы\\одной длины} & \makecell{все работы\\одной длины} \\
        \hline
        \makecell{Связность                                                                         \\графа работ}                  & 1.66                                       & 5.55   & 3.64                   & 3.83                   \\
        \hline
        \makecell{Количество                                                                        \\процессоров}                 & 2                                          & 10     & 256                    & 4096                   \\
        \hline
        \makecell{Полный граф                                                                       \\связности\\процессоров}      & да                                         & да     & нет                    & да                     \\
        \hline
        \makecell{Одинаковые задержки                                                               \\на передачу данных} & да                                         & да     & да                     & нет                    \\
    \end{tabular}

\end{table}

\end{document}